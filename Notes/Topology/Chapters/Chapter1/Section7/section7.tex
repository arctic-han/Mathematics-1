%----------------------------------------------------------------------------------------
%	SECTION 1.5
%----------------------------------------------------------------------------------------

\section{Continuous Functions.}

\begin{definition}
    Let $X$ and  $Y$ be topological spaces. We say that a mapping $f:X
    \rightarrow Y$ is  \textbf{continuous} if for each open set $V$ in  $Y$,
    $f^{-1}(V)$ is open in $X$.
\end{definition}

Now if $f:X \rightarrow Y$ is continuous, the for every open set  $V$ of  $Y$,
$f^{-1}(V)$ is open in  $X$. Now suppose that  $\Bc$ is a basis of  $Y$, then
$V=\bingcup{B_{\alpha}}$, hence
$f^{-1}(\bingcup{B_{\alpha}})=\bingcup{f^{-1}{B_{\alpha}}}$, which is open in
$X$, thus  $B_{\alpha}$ must also be open in  $X$. 

Similarly, if $\Sc$ is a subbasis of  $Y$, then for any basis element  $B$ of
$Y$,  $B=\bigcap_{i=1}^{n}{S_i}$, which then implies that
$f^{-1}(B)=\bigcap_{i=1}^{n}{f^{-1}(S_i)}$, thus  $S_i$ is also open in  $X$ for
 $1 \leq i \leq n$.

\begin{example}
    \begin{enumerate}[label=(\arabic*)]
        \item Let $f:\R \rightarrow \R$ be a continuous realvalued function.
            THen for each open interval  $I \subseteq \R$,  $f^{-1}(I)$ is an
            open interval in  $\R$, so take  $x_0 \in \R$ and $\epsilon>0$, and
            let $I=(f(x_0)-\epsilon,f(x_0)+\epsilon)$, then since $ x_0 \in
            f^{-1}(I)$, there is a basis $(a,b) \subseteq f^{-1}(I)$ about
            $x_0$. Then take  $\delta=\min\{x_0-a,x_0-b\}$, then $x \in (a,b)$
            whenever $0<|x-x_0|<\delta$, and we get that $f(x) \in I$, that is,
            $|f(x)-f(x_0)|<\epsilon$. This is the definition of continuity
            defined in the real analysis. We can prove that the converse holds
            also.

            If $f:\R \rightarrow \R$ is continuous at a point $x_0$, then for
            every $\epsilon>0$, there is a  $\delta>0$ such that
            $|f(x)-f(x_0)|<\epsilon$ whenever $0<|x-x_0|<\delta$. Then we notice
            that $x$ and  $ x_0$ are distinct, furthermore, $x_0-\delta<x
            <x_0+\delta$, hence $x \in (x_0-\delta,x_0+\delta)$ implies that
            $f(x) \in (f(x_0)-\epsilon,f(x_0)+\epsilon)$. Letting
            $V_{\delta}(x_0)=(x_0-\delta,x_0+\delta) $ and
            $V_{\epsilon}(f(x_0))=(f(x_0)-\epsilon,f(x_0)+\epsilon)$, we have
            that whenever  $x \in V_{\delta}(x_0)$, then $f(x) \in
            V_{\epsilon}(f(x_0)) \subseteq f^{-1}(V_{\delta}(x_0))$. And so the
            topological definition of continuity is equivialent to the real
            analytic definition of continuity.

        \item Let $f:\R \rightarrow \R_l$ be defined such that  $f(x)=x$ for all
            $x \in \R$. Take  $[a,b) \subseteq \R_l$, we have that
            $f^{-1}([a,b))=[a,b)$, which is not open in  $\R$  (under the
            standard topology), hence $f$ is not continuous. However, the map
            $g:\R_l \rightarrow \R$ defined the same way is continuous since
            $g^{-1}((a,b))$ is open in  $\R_l$.
    \end{enumerate}
\end{example} 
