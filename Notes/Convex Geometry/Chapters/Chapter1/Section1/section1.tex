%----------------------------------------------------------------------------------------
%	SECTION X.X
%----------------------------------------------------------------------------------------

\section{Definitions and Examples}

\begin{definition}
    We call the $d$-dimensional vector space  $\R^d$ the  \textbf{Euclidean space}, and it is 
    the set of all vectors (also called points) $(x_1, \dots, x_d)$, where $x_i \in \R$ 
    for  $1 \leq i \leq d$. We define the \textbf{Euclidean norm} to be the function 
    $||\cdot||: \R^d \rightarrow \R$ such that for $x=(x_1, \dots, x_m) \in \R^d$, $||x||=\sqrt{x_1^2+\dots+x_m^2}$; 
    and we define the distance of of two points $x,y \in \R^d$ to be the function 
    $\Delta: \R^d \cross \R^d \rightarrow \R$ such that $\Delta(x,y)=||x-y||$.
\end{definition}

\begin{definition}
    Let $x_1,x_2, \dots, x_m$ be points in $\R^d$. We call a point  $x \in \R^d$, of the form 
   $x=\sum_{i=1}^{m}{\alpha_ix_i}$, with  $\sum{\alpha_i}=1$, a \textbf{convex combination} of $x$. We 
   call the set of all convex combinations of a subset  $A \subseteq \R^d$ the 
   \textbf{convex hull} of $A$, and denote it  $\conv{A}$.
\end{definition}

\begin{definition}
    Let $x,y  \in \R^d$. We call the set  $[x,y]=\{\alpha x+(1-\alpha)y: 0 \leq \alpha \leq 1\}$ of all 
    convex combinations of $x$ and  $y$ an \textbf{interval} with \textbf{endpoints}  $x, y$.

    We call a set  $A \subseteq \R^d$ \textbf{convex} if  whenever $x,y \in A$,  $[x,y] \in A$.
\end{definition}

\begin{example}
   The empty set, regular polyhedra, and open balls in $\R^d$ are all convex. 		
\end{example} 

\begin{figure}
    \centering
    \includegraphics[scale = 0.2]{Figures/convexSets.png}
    \caption{Two convex set, and a non-convex set}.
    \label{fig1.1}
\end{figure}

\begin{lemma}\label{1.1.1}
    The convex hull of a convex set is convex.
\end{lemma}
\begin{proof}
    Let $A \subseteq \R^d$, and let $x,y \in \conv{A}$. Then by definition the set of all convex 
    combinations of $x$ and  $y$ is in  $\conv{A}$, thus  $[x,y] \in \conv{A}$
\end{proof}

\begin{definition}
    Let $c_1, \dots, c_m \in \R^d$ and let $\beta_1, \dots, \beta_m \in \R$; we call the set 
    $A=\{x \in \R^d: \langle{c_i,x}\rangle \leq \beta_i \text{, for } 1 \leq i \leq m\}$ a 
    \textbf{ regular polyhedron}.
\end{definition}

\begin{figure} 
    \centering
    \includegraphics[scale = 0.3]{Figures/regularPolyhedra.png}
    \caption{Regular polyhedra in $\R^2$ and  $\R^3$}
    \fig{fig1.2}
\end{figure}

\begin{lemma}\label{1.1.2}
    Regular polyhedra in $\R^d$ are convex.
\end{lemma}
\begin{proof}
    Let $A \subseteq \R^d$ be a regular polyhedron and let  $x,y \in A$. Then 
    $\langle{c_i,x}\rangle, \langle{c_i,y}\rangle \leq \beta_i$ for $c_i \in \R^d$, $\beta_i \in \R$ 
    for  $1 \leq i \leq m$.Then by the scalar linearity of the innerproduct,  $\langle{c_i, \alpha x+ (1-\alpha)y} \rangle
    \leq \beta_i$. Thus $[x,y] \in A$.
\end{proof}

\begin{example}
    Let $V=\{v_1, \dots, v_m\} \subseteq \R^d$, $\rho_1, \dots, \rho_m \in \R^d$, define $f:\R^d \rightarrow \R$ by 
    $f(x)=\sum{\rho_ie^{\langle{x,v_i}\rangle}}$, and define  $H: \R^d \rightarrow \R^d$ by 
        \begin{equation*}
            H(x)=\frac{\sum{\rho_iv_ie^{\langle{x,v_i}\rangle}}}{f(x)}
        \end{equation*} 
    Letting $x,y \in \R^d$, and choosing  $0 \leq \alpha \leq 1$, using the scalar linearity of 
    the inner product, we see that:
        \begin{equation*}
            H(\alpha x_(1-\alpha)y)=
            \frac{\sum{\rho_iv_ie^{\langle{\alpha x+(1-\alpha)y,v_i}\rangle}}}
            {f(\alpha x+(1-\alpha)y)}
        \end{equation*}
        so $H(\R^d) \in \conv{V}$, which also implies that $H(\R^d)$ inherits 
        convexity.
\end{example} 

\begin{example}
    Consider the function $H$ in the example above. Let $y \in \con{V}$, and choose 
    $x=(1-\alpha_m)y' \in \R^d$ such that $||H(x)||<\frac{\epsilon}{2}$; for $\epsilon>0$. Then since 
    $y<x$,  $||H(y)||<||H(x)||<\frac{\epsilon}{2}$. Thus we have that $||H(x)-y|| \leq 
    ||H(x)||+||y||<\frac{\epsilon}{2}+\frac{\epsilon}{2}=\epsilon$.

    Moreover, we see that $H$ is 1-1, for suppose there is no nonzero vector  $c \in \R^d$ 
    such that  $\langle{c,v_i}\rangle=\alpha$ for some $\alpha$. If  $H(x)=H(y)$, then 
    $\langle{x,v_i}\rangle=\langle{y,v_i}\rangle=\alpha$, and so by our supposition, $x=y=0$.
\end{example}

\begin{example}
    Let $q_1,q_2: \R^n \rightarrow \R$ be quadratic forms, and let $S^{n-1}$ be the unit 
    ball in $\R^n$. Define $T:\R^n \rightarrow \R^2$ by  $T(x)=(q_1(x),q_2(x))$. Then 
    $T(S^{n-1})$ is convex in  $R^2$, provided that  $n>2$.
\end{example} 

\begin{theorem}[The Shur-Horn Theorem]\label{1.1.3}
    Let $A=(\alpha_{ij})$ be an  $n \cross n$ matrix, with diagonal $\diag{A}=(\alpha_{11}, \dots \alpha_{nn})$
    and let  $\lambda_1, \dots, \lambda_n \in \R$. Let $X \subseteq \R$ be the set of all diagonals of  $n \times n$ 
    \symmetric matrices with eigenvalues $\lambda_1, \dots, \lambda_n$, Then $X$ is a convex set. 
    Morever, if  $l=(\lambda_1, \dots, \lambda_n)$ is the vecotr of eigenvalues, and  $\sigma$ is 
    a permutation on  $\{1, \dots, n\}$, then $X=\conv{\{l^{\sigma}: 1 \leq i \leq n\}}$.
\end{theorem}
