\documentclass[12pt]{book}

\usepackage[margin=1in]{geometry}
\usepackage{amsmath,amsfonts,amsthm,amssymb,graphicx,mathtools,tikz,hyperref}
\usetikzlibrary{positioning}
\newcommand{\n}{\mathbb{N}}
\newcommand{\z}{\mathbb{Z}}
\newcommand{\q}{\mathbb{Q}}
\newcommand{\cx}{\mathbb{C}}
\newcommand{\real}{\mathbb{R}}
\newcommand{\field}{\mathbb{F}}
\newcommand{\ita}[1]{\textit{#1}}
\newcommand{\com}[2]{#1\backslash#2}
\newcommand{\oneton}{\{1,2,3,...,n\}}
\newcommand\idea[1]{\begin{gather*}#1\end{gather*}}
\newcommand\ef{\ita{f} }
\newcommand\eff{\ita{f}}
\newcommand\proofs[1]{\begin{proof}#1\end{proof}}
\newcommand\inv[1]{#1^{-1}}
\newcommand\setb[1]{\{#1\}}
\newcommand\en{\ita{n }}
\newcommand{\vbrack}[1]{\langle #1\rangle}

\theoremstyle{plain}
\newtheorem{theorem}{Theorem}[section]
\newtheorem{lemma}[theorem]{Lemma}
\newtheorem{proposition}[theorem]{Proposition}
\newtheorem*{corollary}{Corollary}

\theoremstyle{definition}
\newtheorem{definition}{Definition}[section]
\newtheorem{conjecture}{Conjecture}[section]
\newtheorem{exmpple}{Example}[section]

\theoremstyle{remark}
\newtheorem*{remark}{Remark}
\newtheorem{claim}{Claim}
\newtheorem*{note}{Note}
 \date{}


\title{Mathematics Notes}
\author{Alec Zabel-Mena \\ Text -Herstein, I.N., "Topics in Algebra" 2nd edition (Wiley). (1964). }

\begin{document}

\maketitle

% Chapter Template

\chapter{Mappings and The Integers} % Main chapter title

\label{Chapter 1} % Change X to a consecutive number; for referencing this chapter elsewhere, use \ref{ChapterX}

%% to include section files use the \input{} command.

%----------------------------------------------------------------------------------------
%	SECTION 1
%----------------------------------------------------------------------------------------

\section{Mappings}

\hspace{5mm}One topic of great importance in the whole of mathematics, especially in tat of Algebra, Calculus, Topology; is the notion of a "mapping" or "function". A mapping can informally be thought of as being a rule that takes an element from a specified set $S$ and sends it to a set $T$. We will however, require a formal definition of such an object if we want to be able to do mathematics with it.

\begin{definition}
    Let $S \neq \emptyset$ and $T \neq \emptyset$ be sets. A \ita{Mapping}, $M:S \rightarrow T$ is the set $M \subseteq S \times T$; such that $\forall s \in S$, there exists a unique $t \in T$ shuch that $(s,t) \in M$
\end{definition}

If we let $\sigma :S \rightarrow T$ be a mapping, we can also write $S\xrightarrow{\sigma} T$, or if $(s,t) \in \sigma$ is known, we can write $\sigma :s \rightarrow t$.We represent the latter by using the notation "$\sigma(s)=t$" (however some algebraists will write $(s)\sigma =t$ or $s\sigma = t$, we however will use $\sigma(s)=t$). In fact, it turns out that, though precise, our current definition is a bit clunky; so we can introduce an equivalent definition which allows use to make use of our prefered notation:

\begin{definition}
    Let $S \neq \emptyset$ and $T \neq \emptyset$ be sets. A \ita{Mapping}, $M:S \rightarrow T$ is the set $M \subseteq S \times T$; such that:
        \begin{enumerate}
            \item If $s \in S$, then  $\exists t \in T$ such that $M(s)=t$
            \item If $s \in S$ and $t_1,t_2 \in T$ $M(s)=t_1$ and $M(s)=t_2$ then $t_1=t_2$
        \end{enumerate}
\end{definition}

\begin{definition}
    Let $M:S \rightarrow T$ be a mapping. The \ita{Image} of $S$ under $M$ is the set
        \begin{equation}
            M(S)=\{M(s):s \in S\}
        \end{equation}
\end{definition}

Using the definition of an image, we can further define for $s \in S$, $M(s)=t$ to be the \ita{image} of $s$ under $M$


\end{document}
