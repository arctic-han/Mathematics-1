%----------------------------------------------------------------------------------------
%	SECTION X.X
%----------------------------------------------------------------------------------------

\section{The Well Ordering Principle.}

So far we have the set of all natrual numbers $\N$, the set of all integers $\Z$, the rationals $\Q$ and the reals $\R$. 
Now $\N$ is speacial from these, as it has a ``minimal'' element. We clarify this below.

\begin{definition}
    An element $x \in \R$ is a \textbf{least element} in a set $E \subseteq \R$ if and onlt if $x \in E$ and $x \leq \alpha$ 
    for all $\alpha \in E$.
\end{definition}

\begin{postulate}[The Well Ordering Principle.]
  Every nonempty subset of $\N$ has a least element.
\end{postulate}

Now this propertie does not apply to $\Z$, $\Q$, and $\R$; if one takes the subset $\Q$, one sees immediately that $\Q$ has 
no least element. Another thing is that the well ordering principle implies the principle of mathematical induction.

\begin{theorem}
  Suppose that for each $n \in \N$, that $A(n)$ is a propostision such that:
      \begin{enumerate}[label=(\arabic*)]
        \item $A(1)$ is true.

        \item For every $k \in \N$ for which $A(n)$ is true, then $A(k+1)$ is also true.
      \end{enumerate}
    Then $A(n)$ is true for all $n \in \N$/
\end{theorem}
\begin{proof}
  Suppose there is some $n$ for whcih $A(n)$ is false. Then the set $E=\{n \in \N: A(n) \text{ is false.}\} \neq \emptyset$. 
  Then by the well ordering principle, $E$ has a least element $x$. Then $A(x)$ is false. However, since $A(1)$ is true, 
  $x \neq 1$, then $x-1 \in \N$ and $x-1<x$, since $x$ is the least element, then $A(x-1)$ is true. By the second condition, 
  we get that $A(x)$ is true, contradicting that $x \in E$. Therefore, $E$ must be empty, and $A(n)$ is true for all 
  $n \in \N$.
\end{proof}

\begin{example}
  Show that $\sum_{k=1}^{n} (3k-1)(3k+2)=3n^3+6n^2+n$.
\end{example}
\begin{proof}
  For $k=1$ we see that $(3(1)-1)(3(1)+2)=10=2(1)^3+6(1)^2+1$.

  Now suppose that the proposition is true for $n \geq 1$. Then:
      \begin{align*}
        \sum_{k=1}^{n+1} ((3k-1)(3k+2)) &= \sum_{k=1}^{n} ((3k-1)(3k+2))+(3(n+1)-1)(3(n+1)+2) \\
        &= 3n^3+6n^2+n+(3(n+1)-1)(3(n+1)+2) \\
        &= 3(n+1)^3+6(n+1)^2+(n+1)
      \end{align*}
\end{proof}

\begin{remark}
  If $m,n \in \N$, then $m+n \in \N$ and $mn \in \N$. This also implies that $m+n, mn \in \Z$.
\end{remark}

\begin{definition}
  Let $a,b \in \Z$. We call a \textbf{binomial} an expression of the form $(a+b)^n$ for some $n \in \N$.
\end{definition}

We wish to study binomials further.

\begin{definition}
  We define, for $n,k \in \N$ the \textbf{binomial coefficient} $n \choose k$ is defined such that:
    \begin{enumerate}[label=(\arabic*)]
      \item ${0 \choose 0} =1$

      \item ${n \choose k}=\frac{n!}{(n-k!)k!}$.
    \end{enumerate}
\end{definition}

The binomial coefficient is itself a natural number, and can be visualized via pascal's triangle.

\begin{lemma}\label{lemma1.2.2}
  ${n \choose k}+{n \choose k-1}={n+1 \choose k}$ For $n,k \in \N$ and $1 \leq k \leq n$.
\end{lemma}
\begin{proof}
  \begin{align}
    {n \choose k}+{n \choose k-1} &= \frac{n!}{(n-k)!k!}+\frac{n!}{(n-k+1)!(k-1)!} \\
                                  &=  n!(\frac{1}{(n-k)!k!}+\frac{1}{(n-k-1)!(k-1)!}) \\
                                  &= n!(\frac{1}{(k-1)!}(\frac{1}{(n-k)!k}+\frac{1}{(n-k-1)!}))\\
                                  &=\frac{n!}{(k-1)!(n-k)!}(\frac{1}{k}+\frac{1}{n-k+1})\\
                                  &= \frac{(n+1)!}{(n-k+1)!k!} \\
                                  &= {n+1 \choose k}
  \end{align}
\end{proof}

\begin{theorem}[The Binomial Theorem]
  If $a,b \in \R$, and $n \in \N$, then:
    \begin{equation}\label{equation1.8}
      (a+b)^n=\sum_{k=0}^{n} {n \choose k}a^{n-k}b^k
    \end{equation}
\end{theorem}
\begin{proof}
  We use induction in the proof. We see that for $n=1$, $(a+b)^1=a+b={1 \choose 0}a^{1-0}+{1 \choose 1}b^1$. Now suppose that 
  the theorem is true for all $n \geq 1$, that is $      (a+b)^n=\sum_{k=0}^{n} {n \choose k}a^{n-k}b^k$; and now consider $n+1$.Then:
    \begin{align}
      (a+b)^{n+1} &= (a+b)(a+b)^n \\
                  &= (a+b)(\sum_{k=0}^{n} {n \choose k}a^{n-k}b^k) \\
    \end{align}
by the distributive law:
    \begin{align}
                  &= a\sum_{k=0}^{n} {n \choose k}a^{n-k}b^k)+b\sum_{k=0}^{n} {n \choose k}a^{n-k}b^k)
                  &=\sum_{k=0}^{n} {n \choose k}a^{n-k+1}b^k)+\sum_{k=0}^{n} {n \choose k}a^{n-k}b^{k+1}).
    \end{align}
Listing the terms we get ${n \choose 0}a^{n+1}+{n \choose 1}a^{n}b+{n \choose 2}a^{n-1}b^2+ \dots + {n \choose n+1}ab^n+{n+1 \choose n+1}b^n$; 
adding like terms, and by lemma \ref{lemma1.2.2} we get: ${n+1 \choose 0}a^{n+1}+{n+1 \choose 1}a^{n}b+{n+1 \choose 2}a^{n}b^2
+ \dots + {n+1 \choose n}ab^n+{n+1 \choose n+1}b^n = \sum_{k=0}^{n+1} {n+1 \choose k}a^{n-k+1}b^k$.
\end{proof}

\begin{HW}
  Exercises: 1,2,3,5,6 on page 17. These exercises practice the principle of mathematical induction.
\end{HW}
