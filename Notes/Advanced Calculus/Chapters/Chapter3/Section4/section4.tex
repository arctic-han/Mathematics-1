%----------------------------------------------------------------------------------------
%	SECTION 3.3
%----------------------------------------------------------------------------------------

\section{Uniformity.}

We say a function  $f:I \rightarrow \R$ is continous if  $\lim{f}=f(a)$ as  $x \rightarrow a$. 
That is for any  $\epsilon>0$, there is a  $\delta(\epsilon)>0$ such that  $|x-a|<\delta$ implies 
that  $|f(x)-f(a)|<\epsilon$. Here,  $\delta$ depends on  $\epsilon$, however, $\delta$ usually also depends
on the domain  $I$ of  $f$,  $f$ istself, and even  $a$. If  $f$ is the constan function, then 
$\delta$ does not depend on  $\epsilon$, as  $|f(x)-f(a)|=0$. The same is true for  $a$.

\begin{definition}
    Let $E \subseteq \R$ be nonempty, and let $f:E \rightarrow \R$ be a realvalued function. We call 
    $f$ \textbf{uniformly continous} on $E$ if and only if for every $\epsilon>0$, there is a
    $\delta>0$ such that $|x-a|<\delta$ and  $a,x \in E$ imply that $|f(x)-f(a)|<\epsilon$.
\end{definition}

That is to say, a function is uniformly continous if and only if $\delta$ depends only 
on  $\epsilon$,  $E$, or  $f$.

\begin{example}
    $f(x)=x^2$ is uniformly continuous on $(0,1)$.
\end{example} 
\begin{proof}
    Given $\epsilon>0$, let  $x,a \in (0,1)$. Then $|x^2-a^2|=|x-a||x+a| \leq  (|x|+|a|)|x-a|<2|x-a|$. 
    Then choose $\delta=\frac{\epsilon}{2}$, then we have $|x-a|<\delta$, for  $x,a \in (0,1)$ implies 
    $|f(x)-f(a)|<\epsilon$. So $f(x)=x^2$ is uniformly continuous on  $(0,1)$. If we choose $E=(0,2)$ for  $f(x)=x^2$, then we 
    require  $\delta=\frac{\epsilon}{4}$. So $f(x)=x^2$ is uniformly continuous on any bounded domain.
\end{proof}

\begin{example}
    $f(x)=x^2$ is not uniformly continuous on  $\R$.		
\end{example} 
\begin{proof}
    For $\epsilon>0$, and choosing $x,a \in \R$, Suppose that  $f$ is uniformly continous on  $\R$. Then 
    there is a  $\delta>0$ such that  $|x-a|<\delta$ implies  $|f(x)-f(a)|<\epsilon$. Choose $\epsilon=1$, then 
    there is a corresponding $\delta$ for which $|f(x)-f(a)|<1$. Letting  $x=a+\frac{\delta}{2}$, 
    we have that $|x-a|<\delta$, and  $|x^2-a^2|=|\delta+\frac{\delta^2}{4}|>1$, if 
    we suppose that $a>1$. So  $a$ can be as large as possible, and this contradicts the assumption 
    that  $|f(x)-f(a)|<1$. So  $f$ is not uniformly continous on  $\R$.  (In general, we can show 
    that $f$ is not uniformly continous on  $\R$ by using the Archimedian principle for any  $n \in \N$).
\end{proof}

\begin{example}
    $f(x)=\frac{1}{x}$ is continous on $(0,1)$, but not uniformly continuous on  $(0,1)$.		
\end{example} 

\begin{lemma}\label{3.4.1}
    Suppose that $E \subseteq \R$ and  $f:E \rightarrow \R$ is unifomrly continous. If  $\{x_n\} \subseteq E$ is 
    Cauchy, then  $\{f(x_n)\} \subseteq \R$ is also Cauchy.
\end{lemma}
\begin{proof}
    Suppose that $f$ is uniformly continuous. Then for evey  $\epsilon>0$, there is a 
    $\delta>0$ such that  $|x-a|<\delta$ for  $x,a \in E$ implies  $|f(x)-f(a)|<\epsilon$. We 
    also have that for this  $\delta$, there is an  $N>0$ such that for  $m,n \geq N$,  $|x_n-x_m|<\delta$.
    Then  $|f(x_n)-f(x_m)|<\epsilon$ whenever  $x_n,x_m \in E$. Thus we have that  $\{f(x_n)\}$ is 
    a Cauchy sequence.
\end{proof}

\begin{theorem}\label{3.4.1}
    Suppose that $I$ is a closed, bounded interval. If  $f:I \rightarrow \R$ is contnuous on  $I$; then 
     $f$ is uniformly continous on  $I$.
\end{theorem}
\begin{proof}
    Suppose that $f:I \rightarrow \R$ is continous, but not uniformly continous on  $I$. 
    Then for every $\epsilon>0$, there is a  $\delta>0$ such that  $|x-a|<\delta$ implies  $|f(x)-f(a)|<\epsilon$.
    On the other hand, we have that there is some  $\epsilon>0$ for which given any  $\delta>0$,  
    $|x-a|<\delta$, with $x,a \in I$ implies  $|f(x)-f(a)| \geq \epsilon$.

    Let $\epsilon_0>0$, and for every  $\delta>0$, choose  $x_n,y_n \in I$ such that $|x_n-y_n|<\delta$, but 
    $|f(x_n)-f(y_n)| \geq \epsilon_0$. Now since $I$ is bounded, then so are the sequences  $\{x_n\}$ and  $\{y_n\}$; 
    so by thr Bolzanno-Weierstrass theorem, they have convergent subsequences. By the sequential criterion, 
    $f(x_n), f(y_n)$ converge to the same point, which contradicts our assumption. Therefore, 
     $f$ must be uniformly continuous.
\end{proof}

\begin{theorem}\label{3.4.3}
    Let $(a,b)$ be a bounded open, nondegenerate interval, and let  $f:(a,b) \rightarrow \R$ be a realvalued function.
    Then  $f$ is uniformly continous on  $(a,b)$ if and only if  $f$ is continous on $[a,b]$.
\end{theorem}
\begin{proof}
    Suppose that $f$ is unifomly continuous on  $[a,b]$, then clearly it is unifomly 
    continuous on $(a,b)$. Now suppose that  $f$ is uniformly continous on  $(a,b)$. Define 
    $f(a)$ and  $f(b)$ continuously, that is define  $f(a)$ and  $f(b)$ sicht that 
    $f(x^+)=f(a)$ and  $f(x^-)=f(b)$. Pick  $\{x_n\} \subseteq (a,b)$ with $\lim{x_n}=a$, and 
    $\{x_n\}$ is Cauchy. Then  $f$ is uniformly continous and  $\{f(x_n)\}$ is also Cauchy. So 
    define $lim{f}=f(a)$. Then choose $\{y_n\}$ corresponding to the upper criteria. Then we 
    have that  $x-y \rightarrow 0$. Now given  $\epsilon>0$, choose  $\delta>0$ such that  $|x_n-y_n|<\delta$ 
    implies  $|f(x)-f(y)|<\epsilon$. Then there is an  $N>0$ such that when $n \geq N$,  
    $|x_n-y_n|<\delta$, hence  $|f(x_n)-f(y_n)|<\epsilon$. Then $\lim{f(x_n)}=\lim{f(y_n)}$.

    Then by the sequential criterion, we have that  $f(a)=f(x^+)$ and  $f(b)=f(x^-)$. Thus 
    we have a continous extention of  $f$ onto  $[a,b]$.
\end{proof}

\begin{HW} 
    Exercises $1$, $2$, $3$ and  $4$ on page  $83$.
\end{HW}
