%----------------------------------------------------------------------------------------
%	SECTION 3.3
%----------------------------------------------------------------------------------------

\section{Continuity.}

In general, we can say that realvalued function $f$ is \textbf{continuous} at  
$a$ if  $f$ is defined at  $a$, and  $\lim{f(x)}=f(a)$ as  $x \rightarrow a$. We give the 
formal definition below.

\begin{definition}
    Let $E \subseteq \R$ be nonemoty, and let  $f:E \rightarrow \R$ be a realvalued 
    funtion. $f$ is said to be  \textbf{continuous} at a point  $a \in E$, if 
    given  $\epsilon>0$, there is a  $\delta(\epsilon)>0$ such that  $|x-a|<\delta$, 
    and $x \in E$ imply  $|f(x)-f(a)|<\epsilon$. $f$ is said to be \textbf{continuous}, on  $E$ \
    if and only if  $f$ is continuous at  $x$ for all  $x \in E$.
\end{definition}

\begin{remark} 
    we have that a realvalued function $f$ is continuous at a point $a$ on its 
    domain if and only if $\lim{f}=f(a)$ as  $x \rightarrow a$. We see that $\lim{f(x)}=
    f(\lim{x})=f(a)$ as $x \rightarrow a$. That is to say, if you can commute  $f$ and 
     $\lim$, then  $f$ is a continuous function.
\end{remark}

\begin{theorem}\label{3.3.1}
    Suppose that $E \subseteq \R$ is nonempty, and that  $a \in E$, and  $f:E \rightarrow \R$ 
    is a realvalued function. THen the following are equivalent:
         \begin{enumerate}[label=(\arabic*)]
             \item $f$ is continuous at $a$.

             \item If $x_n \rightarrow a$, and $x_n \in E$, then $f(x_n) \rightarrow f(a)$ as $n \rightarrow a$.
        \end{enumerate}
\end{theorem}
\begin{proof}
    We apply the sequential characterization of limits of functions.
\end{proof}

\begin{theorem}\label{3.3.2}
    Let $E \subseteq \R$ be nonempty, and let  $f : E \rightarrow \R$ and $g:E \rightarrow \R$ be realvalued 
    functions. If  $f$ and  $g$ are continuous at a point  $a \in \R$, then so 
    is  $f+g$,  $fg$,  $\alpha f$ for  $\alpha \in \R$. Moreover, $f/g$ is continuous at  
    $a \in E$  when $g(a) \neq 0$  (specifically, when $g(x) \neq 0$ for all  $a \in E$).
\end{theorem}
We omit the proof. What this theorem says algebraically, is the the set of all 
contnuous functions over a domain $E$ form an algebra. Now consider for realvalued 
functions continuous at a point $a \in E$, then
    \begin{align*}
        f^+(x) &= \frac{f(x)+|f(x)|}{2} = \max\{f(x),0\} \\
        f^-(x) &= \frac{|f(x)|-f(x)}{2} = \max\{-f(x),0\}	\\
    \end{align*}
Then notice that $f=f^+-f^-$, and so any function is the difference of two non
negative functions, moreover $|f|=f^++f^-$. Moreover, all polynomial functions 
are continuous over $\R$.

\begin{definition}
    We define the composition of functions as the binary operation $\circ$ such 
    that if $f:A \rightarrow B$ are fubctions and  $g: B \rightarrow C$, then $f \circ g: A \rightarrow C$ is 
    the function defined by  $f \circ g(x)=f(g(x))$.
\end{definition}
It is well known that $\circ$ is not commutative, i.e.  $f \circ g \neq g \circ f$.

\begin{theorem}\label{3.3.3}
    Suppose that $A, B \subseteq \R$ and that $f:A \rightarrow \R$ and $g: B \rightarrow \R$ are realvalued 
    functions, with $F(A) \subseteq B$. Then if $A=I \backslash \{a\}$ where $I$ is a nondegenerate 
    interval, that either contains $a$, or has $a$ as one of its endpoints. 
    If $L=\lim{f}$ as $x \rightarrow a$ for $x \in I$ exists and belongs to $B$, and if  $g$
    is continuous at $L \in B$, then  $\lim{g \circ f}=g(\lim{f})$ as 
    $x \rightarrow a$  for $x \in I$.
\end{theorem}
\begin{proof}
    Let $\{x_n\} \subseteq I \backslash \{a\}$, such that $x_n \rightarrow a$ as $n \rightarrow \infty$. Then if $f(A) \subseteq B$ with 
    $f(x_n) \in B$, then  $L=\lim{f}$ as  $x \rightarrow a$, then by the sequential characterization 
    of limits of functions, for  $x \in I$, so $f(x_n) \rightarrow L$ as $n \rightarrow \infty$. This implies 
    that $g(fx_n)) \rightarrow g(L)$ as  $n \rightarrow \infty$, and we are done.
\end{proof}

\begin{corollary}
    If $f$ is continuous at $a \in A$, and $g$ is continuous at $f(a) \in B$, 
    then $g \circ f$ is continuous at $a \in A$.
\end{corollary}

\begin{definition}
    Let $E \subseteq \R$ be a nonempty subset of $\R$. A realvalued function $f:E \rightarrow \R$ 
    is said to be \textbf{bounded} if and only if there is an $M \in \R$ sucht that 
    $|f| \leq M$ for all  $x \in E$.
\end{definition}

\begin{theorem}[The Extreme Value Theorem]\label{3.3.4}
    If $I$ is closed, bounded interval, and  $f:I \rightarrow \R$ is continuous 
    on  $I$,then  $f$ is bounded on  $I$; moreover, if $M=\sup{f(I)}$ and 
    $m=\inf{f(I)}$, then there exists points $x_m$ and  $x_M$ such that
    $f(x_M)=M$ and  $f(x_m)=m$.
\end{theorem}
\begin{proof}
    Suppose that $f$ is not bounded on  $I$, then there is some  $x_n \in I$ 
    such that $|f(x_n)|>n$ for every $n \in \N$. Now $I is bounded$, hence so 
    is the sequence $\{x_n\}$ of points of  $I$. Thus, by the Bolzano Weierstrass 
    theorem,  $\{x_n\}$ has a convegernt subsequence, $x_{n_k} \rightarrow a$ 
    as $k \rightarrow \infty$. Now since  $I$ is closed, it contains all the 
    limit points, so  $a \in I$. Since  $f(a) \in I$, and since  $f$ is continuous, 
    then  $\lim{|f|}=\lim{|f(x_{n_k})|}=\infty$ as  $k \rightarrow \infty$, which 
    is a contradiction. Hence $f$ is bounded on  $I$.

    Now because  $f$ is bounded on $I$, then $M,m$ are finite. We show that 
    there is an $xM \in I$ with $f(x_M)=M$. Suppose that  $f<M$ for all  $x \in I$, 
    define  $g:I \rightarrow \R$ by
        \begin{equation*}
            g(x)=\frac{1}{M-f(x)}
        \end{equation*}
    We have that $g$ is contininous, and bounded on $I$. Then there is a  $C>0$ 
    such that  $|g| \leq C$, so $g(x) \leq C$ for all  $x \in I$. Then by definition 
    of  $g$ we have that  $f \leq M-\frac{1}{C}<M$ for all $x \in I$, so  $M-\frac{1}{C}$ 
    is an upperbound of $f$ over I which is less than  $\sup{f(I)}$, a contradiction. 
    Thus  $f(x_M)=M$ for some  $x_M \in I$. Similarly, we get that  $f(x_m)=m$.
\end{proof}

We can prove the second statement alternatively by noting that $M=\sup{f(I)}$, 
so for every  $n \in \N$, there is an  $x_n \in I$ such that $M-\frac{1}{n}<f(x_n)
\leq M$, then the sequence $\{x_n\} \subseteq I$ has a convergent subsequence $x_{n_k} \rightarrow a$ by 
the Bolzano Weierstrass theorem. Since $I$ is closed, then we get that  $a \in I$, 
and since $f$ is continuous, then by the comparison theorem
    \begin{equation*}
        M-\frac{1}{n_k}<f(x_{n_k}) \leq M
    \end{equation*}
Then $M \leq f(a) \leq M$, then $f(a)=M$. The proof is also analogous for $m=\inf{f(I)}$.

The dowside of this proof for the second part of the extreme value theorem is that 
it repeats the proof of the first part; that aside, the formal proof given also 
illustrates the relation between the extreme values of $f$, and the extreme values 
of $g$.

For the extreme value theorem to work, the interval $I$ must be closed, and 
bounded. That is it does not work for intervals of the type  $(a,b)$, $(a, \infty)$ and 
$[a,\infty)$, or  $(-\infty, b)$ and  $(-\infty,b]$. $f$ must also be a continous 
function. Remove any one of those properties, and the theorem falls apart.
