%----------------------------------------------------------------------------------------
%	SECTION X.X
%----------------------------------------------------------------------------------------

\section{Limits of Sequences.}

\begin{definition}
    A \textbf{sequence} is a mapping $f:\N \rightarrow X$ for some nonempty set $X$, whose terms are $x_n=f(n)$. We denote a 
    sequence as $\{x_n\}_n \in \N$ or just simply $\{x_n\}$. We say that a sequence is \textbf{real valued} if $X=\R$.
\end{definition}

\begin{definition}
    A sequence of real numbers $\{x_n\}$ is said to \textbf{converge} to a real number $a \in \R$ if and only if for every 
    $\epsilon>0$, there is an  $N \in \N$ (in general depending on $\epsilon$) such that for all $n \geq N$,  $|x_n-a|<\epsilon$. 
    We say that $\{x_n\}$  (or $x_n$) \textbf{converges} to $a$, and we write:
        \begin{equation}
            \lim_{n \rightarrow \infty} x_n=a		
        \end{equation} 
        or simply, $x_n \rightarrow a$ as  $n \rightarrow \infty$. We call $a$ the \textbf{limit} of the sequence. A sequence 
        which does not cpnverge is said to \textbf{diverge}.
\end{definition}

We can consider $x_n$ as a sequence of ``approximations'' to $a$ and  $\epsilon$ as an upperbound for the ``error'' of 
those approximations.  $N \rightarrow \infty$ as  $\epsilon \rightarrow 0$, thay is $N$ gets larger as  $\epsilon$ gets smaller.

\begin{example }
    Prove that $\lim_{n \rightarrow \infty} \frac{1}{n}=0$
\end{example} 
\begin{proof}
    Let $\epsilon>0$, by the Archimedean principle, there is an  $N \in \N$ such that  $\frac{1}{\epsilon}<N$. Then 
    $\frac{1}{N}<\epsilon$. For all $n \geq N$,  $\frac{1}{n}<\frac{1}{N}<\epsilon$, hence, $|\frac{1}{n}-0|<\epsilon$.
\end{proof}

\begin{example }
    The sequence $\{(-1)^n\}_{n \in \N}$ does not converge		
\end{example} 
\begin{proof}
    Assume that $(-1)^n \rightarrow a$ as $n \rightarrow \infty$. By definition, we have that for any  $\epsilon>0$, there is 
    an  $N \in \N$ such that  $|(-1)^n-a|<\epsilon$ for  $n \geq N$. Choose  $\epsilon=\frac{\q}{2}$. then for $n$ od we have 
    $(-1)^n=-1$, and for  $n$ even,  $(-1)^n=1$. Then  $|(-1)^n-(-1)^{n+1}|=|((-1^n)-a)-((-1)^{n+1})-a|$; by the triangle 
    inequality, we have  $|(-1)^n-(-1)^{n+1}| \leq |(-1)^n-a|+|(-1)^{n+1}-a|<\frac{1}{2}<1$, when $n \geq N$. But 
    $|(-1)^n-(-1)^{n+1}|=2$, a contradiction. Thus  $\{(-1)^n\}$ does not converge.
\end{proof}

\begin{remark}
    A sequence can have at most one limit.
\end{remark}
\begin{proof}
    Suppose it has atleast two limits, that is  $x_n \rightarrow a$ and  $x_n \rightarrow b$ (with $a \neq b$) as  $n \rightarrow \infty$. For  
    $\epsilon>0$ there is an  $N_1$ such that  for all $n \geq N_1$  $|x_n-a| \leq \epsilon$, and there is an  $N_2$ such 
    that $|x_n-b| \leq \epsilon$. Choose  $N=\max{(N_1,N_2)},$ then for  $n \geq N$ we have:
        \begin{align*}
            |x_n-a|<\epsilon && \text{and} && |x_n-b|<\epsilon \\
        \end{align*}  
Then $|a-b|=|(a-x_n)-(b-x_n)| \leq |a-x_n|+|b-x_n|<2\epsilon$ for every $\epsilon$. Now choose $\epsilon=|a-b|/4>0$, then 
 $2\epsilon=|a-b|/ 2$ hence  $|a-b|<|a-b|/2$, which is a contradiction. Then $x_n$ converges to at most one limit.
\end{proof}

\begin{definition}
    A \textbf{subsequence} of a sequence $\{x_n\}_{n \in \N}$ is a sequence of the form  $\{x_{n_k}\}_{k \in \N}$ where  $n_k \in \N$ and  
    $n_1<n_2<\dots$.		
\end{definition}
 
Recall that the sequence $\{\(-1)^n\}$ does not converge, however, suppose we only select even terms and form $\{(-1)^{2n}\}$, 
then this latter sequence converges. Like wise the sequence  $\{\frac{1}{n}\} \rightarrow 0$ rather slowly, but forming the 
subsequence $\{\frac{1}{2^n}\}$, the latter converges rather quickly. So the immediate use of subsequences is in the correction 
of sequences that behave badly (i.e they don't converge), or to make them converge quikly. Now if $n_k=k$, then the subsequence 
is the original sequence, if $n_k>k$, then the subsequence is a \textbf{proper} subsequence.

\begin{remark}
    If $\{x_n\}$ converges to  $a$ and $\{x_{n_k}\}$ is any subsequence of $\{x_n\}$, then  $\{x_{n_k}\}$ also converges 
    to  $a$.
\end{remark} 
\begin{proof}
   For any $\epsilon>0$, there is an  $N>0$ such that for all $n \geq N$,  $|x_n-a|<\epsilon$. Now if  $n_k=k$, we are done, 
   so suppose that $n_k>k$, so for  $k=N$,  $n_k>N$ then $|x_{n_k}-a|< \epsilon$.
\end{proof}

\begin{definition}
    Let $\{x_n\}$ be a sequence of real numbers. Then we say that $\{x_n\}$ is \textbf{bounded above} if there is an $M \in \R$ 
    such that $x_n \leq M$ for all $n \in \N$. Similarly, $\{x_n\}$ is \textbf{bounded below} if there is an $m \in \R$ such 
    that  $m \leq x_n$ for all $n \in \N$. We say $\{x_n\}$ is \textbf{bounded} if it is both bounded above and bounded 
    below
\end{definition}

Now $\{x_n\}$ is bounded if and only if there is some  $C \in \R$ with $C>0$ such that  $|x_n| \leq C$ for all $n \in \N$.

\begin{theorem}
    Every convergent sequence is bounded.
\end{theorem}
\begin{proof}
    Let $\{x_n\}$ converge to $a$. Then for every $\epsilon>0$, there is an  $N>0$ such that for every  $n \geq N$,  
$|x_n-a|<\epsilon$. Choosing  $\epsilon=1$, we have  $|x_n-a|<1$, so  $|x_n|=|x_n-a+a| \leq |x_n-a|+|a|<1+|a|$. Now let  
$C=|x_1|+|x_2|+\dots+|x_{N-1}|+|a|+1$. Then we see that $|x_n| \leq C$ for all  $n \in \N$. Thus  $\{x_n\}$ is bounded. 
\end{proof}

The converse of this theorem is not true; take $\{(-1)^n\}$ which is bounded by  $1$, but  $\{(-1)^n\}$ diverges.

\begin{HW}
    Exercises  $1$,  $4$,  $5$,  and $6$ on page  $38$.
\end{HW}
