%----------------------------------------------------------------------------------------
%	SECTION 4.3
%----------------------------------------------------------------------------------------

\section{Monotonic Functions and Inverse Functions.}

\begin{definition}
    Let $E \subseteq \R$ be nonempty, and let  $f:E \rightarrow \R$ be a realvalued function. 
    We say that $f$ is \textbf{monotonically increasing} on $E$ if for $x<y$, $f(x) \leq f(y)$. 
    Similarly, $f$ is \textbf{monotonically decreasing} if $f(y) \leq f(x)$. In either case, we call 
    $f$ a \textbf{monotonic} function.
\end{definition}

\begin{example}
    The function $f(x)=x^2$ is not monotonoic on  $I$, but it is monotonically increasing 
    over $[0,\infty)$ and monotonically decreasing over $(-\infty, 0)$. 
\end{example} 

\begin{theorem}\label{4.4.1}
    Suppose that $a, b \in \R$ , with  $a$ and  $b$ distinct, and let  $f$ be contiuous 
    over  $[a,b]$ and differentiable over  $(a,b)$. Then:
        \begin{enumerate}[label=(\arabic*)]
            \item If $f' \geq 0$, for all  $x \in (a,b)$, then $f$ monotonically increasing 
                on $[a,b]$; respectively, if  $f' \leq 0$,  $f$ is monotonically decreasing.

            \item If $f'=0$, for all  $x \in (a,b)$, then $f$ is constant on  $[a,b]$
        \end{enumerate}
\end{theorem}
\begin{proof}
    Assume without loss of generality that $f' \leq 0$, then for any  $a<x_1<x_2<b$, then 
    $f$ is continuous on  $[x_1,x_2]$ and differentiable on $(x_1,x_2)$, hence by the 
    mean value theorem, then there is a $c \in (x_1,x_2)$ such that $f(x_2)f(x_1)=f'(c)(x_2-x_1)$, 
    hence $f(x_2)-f(x_1) \leq 0$, hence $f$ is monotonically decreasing. The case is analogous 
    for  $f'>0$.

    Now suppose that  $f'=0$,  then again by the mean value theorem, there is a  $c \in (x_1,x_2)$ 
    such that $f(x_2)-f(x_1)=f'(c)(x_2-x_1)$, then $f(x_2)-f(x_2)=0$, for $x_1, x_2 \in (a,b)$ 
    arbitrary therefore, $f$ is constant.
\end{proof}

\begin{remark}
    If $f$ and  $g$ are continuous on a nondegenerate inteval,  $[a,b]$, and differentiable 
    on  $(a,b)$, and  $f'=g'$ for all  $x \in (a,b)$, then $f-g$ is constant.
\end{remark}
\begin{proof}
    We repeat the proof for the constant case using the Cauchy mean value theorem; 
    similarly, we can just apply the previous theorem to the function $f-g$.
\end{proof}

Suppose that $f$ is a realvalued function, who has an inverse function  $f^{-1}$. Then 
the graph of  $f^{-1}$ is symmetric to the graph of  $f$ with respect to the line  $y=x$. Then 
visually,  $f^{-1}$ is as smooth as  $f$. Algebraically, we can apply an operation to the graph 
of $f$ to obtain the graph of  $f^{-1}$, and we can observe that smoothness is preserved, however 
we would like to prove this rigorously.

\begin{theorem}\label{4.4.2}
    If $f$ is a  1-1 continuous function of the interval  $I$ onto $\R$, 
    then  $f$ is strictly monotonic on  $I$, and  $f^{-1}$ is continuous and strictly 
    monotonic on $f(I)$.
\end{theorem}
\begin{proof}
    We have that since $f$ is 1-1, then  $f(x)=f(y)$ implies  $x=y$, hence if  $x<y$ 
    are in $I$, then  either $f(x)<f(y)$ or  $f(x)>f(y)$, now if  $f$ is not strictly monotone, then 
    for some $c \in I$, with  $x<c<y$, we have that either  $f(x)$ is inbetween $f(c)$ and  $f(b)$, 
    or  $f(y)$ is between  $f(x)$ and  $f(c)$, hence, by the intermediate value theorem, there is an  $x_1 \in I$ 
    such that $f(x_1)=f(x)$ or $f(x_1)=f(y)$. Then either $x_1=x$ or $x_2=y$ which contradicts the assumption.

    Now suppos that $f$ is strictly increasing on  $I$, since  $f$ is 1-1 ont  $\R$, then 
    $f^{-1}$ exists on  $f(I)$. Now suppose that there are  $y_1,y_2 \in f(I)$ such that 
    $ y_1<y_2$ but $f^{-1}(y_1) \geq f^{-1}(y_2)$, and since $f$ is increasing, then  $f(x_1) \geq f(x_2)$, which 
    is absurd. We also have that $f(I)$ is an interval for if  $I=[a,b]$, then  $f([a,b])=[f(a)f(b)]$ by 
    the intermediate value theorem.

    Now fix  $y_0 \in f(I)$, and let $\epsilon>0$, since $f^{-1}$ is strictly increasing, on $f(I)$ by 
    the above assumption, if $y_0$ is not a right endpoint of $f(I)$, then  $f^{-1}(y_0)$ is not 
    a  right endpoint of  $I$, then  there is an $0<\epsilon_0<\epsilon$ with $\epsilon+\epsilon_0 \in I$. Let  $\delta=f(x_0+\epsilon_0)-f(x_0)$, 
    and suppose that $0<y-y_0<\delta$, then $y_0<y<y_0+\delta=f(x_0+\epsilon_0)$, then since $f^{-1}$ is strictly 
    increasing, it follows that  $x_0<x<x+\epsilon_0$, hence we have that $0<f^{-1}(y)-f^{-1}(y_0)<\epsilon$, 
    that is $f^{-1}(y_0+)$ exists, similatly, if $y_0$ is not a left endpoint of $I$, then  $f^{-1}(y_0-)$ 
    exists. In both cases, we have that $f^{-1}(y_0\pm)=f(y_0)$. Therefore, $f^{-1}$ is contiuous.
\end{proof}
