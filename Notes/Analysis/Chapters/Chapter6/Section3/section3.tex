%----------------------------------------------------------------------------------------
%	SECTION 1.1
%----------------------------------------------------------------------------------------

\section{The Continuity of Derivatives.}

\begin{theorem}\label{6.3.1}
    Let $f:[a,b] \rightarrow \R$ be differentiable on all of  $[a,b]$, and suppose that 
    $f'(a)<\lambda<f'(b)$. Then there is an  $x \in (a,b)$ such that $f'(x)=\lambda$.
\end{theorem}
\begin{proof}
    Let $g(t)=f(t)-\lambda t$, then  $g'(a)<0$ and  $g'(b)>0$. Then for  $t_1,t_2 \in (a,b)$, 
    $g(t_1)<g(a)$, and $g(b)<g(t_2)$. Then by the extreme value theorem, $g$ attains a maximum at 
    a point  $x \in (t_1,t_2)$, hence $g'(x)=0$, hence  $f'(x)=\lambda$.
\end{proof}

\begin{corollary}
    If $f:[a,b] \rightarrow \R$ is differentiable, then  $f$ cannot have any removable discontinuities, 
    nor jump discontinuities.
\end{corollary}

\begin{remark}
    $f'$ may have infinite discontinuities.
\end{remark}
