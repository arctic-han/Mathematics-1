%----------------------------------------------------------------------------------------
%	SECTION 1.1
%----------------------------------------------------------------------------------------

\section{The Algebraic and Order Properties of $\R$.}

We begin first by examining the algebraic struture of a the real numbers, which we denote by $\R$. The following are called 
the \textbf{field axioms} for $\R$.

\begin{axiom} 
    On the set $\R$ of real numbers, there are two binary operations $+$ and  $\cdot$ called \textbf{addition} and 
    \textbf{multiplication} respectively, such that:
        \begin{enumerate}[label=(\arabic*)]
            \item $(\R,+)$ is an abelian group.

            \item $(\R \backslash \{0\}, \cdot)$ is an abelian group.

            \item  $\cdot$ distributes over $+$; that is for all  $a,b,c \in \R$,  $a \cdot (b+c)= a \cdot b+a \cdot c.$
        \end{enumerate}
\end{axioms}

\begin{theorem}
    \begin{enumerate}[label=(\arabic*)]
        \item If $z,a \in \R$, with $z+a=a$, then  $z=0$.

        \item If $u,b \in \R$ with  $b \neq 0$, and  $ub=b$, then  $u=1$.

        \item For  $a \in \R$,  $a_0=0$.
    \end{enumerate}
\end{theorem}

The proof of this theorem is elementary and easy to reproduce.

\begin{theorem}
    \begin{enumerate}[label=(\arabic*)]
        \item If $a,b \in \R$ with  $a \neq 0$, and  $ab=1$, then  $b=a^{-1}$.

        \item If $ab=0$, then either  $a=0$ or  $b=0$.
    \end{enumerate}
\end{theorem}

Another elementary proof.

\begin{remark} 
    For the real numbers $a,b \in \R$, we define \textbf{subtraction} of  $a$ and  $b$, to be  $a-b=a+(-b)$, similarly, we 
    define \textbf{division} to be $\frac{a}{b}=ab^{-1}$. We also define the \textbf{exponent} recursively to be $a^0=1$,  
    $a^1=a$ and  $a^{n+1}=a^na$ for  $n \geq 1$.We also denote  $\frac{1}{a}=a^{-1}$ and we leave $0^0$  and $\frac{1}{0}$ 
    undefined.
\end{remark}

We call the \textbf{rational numbers} tobe numbers of the form $\frac{a}{b}$ with $a,b \in \Z$ and denote them as $\Q$. There 
are however elements of  $\R$ which are not in  $\Q$; we calll the set of such elements the \textbf{irrational numbers} and 
denonte them by  $\R \backslash \Q$, or  $\Q^*$.

 \begin{theorem}
    There is no rational number $r$ such that $r^2=2$.
\end{theorem}
\begin{proof}
    Suppose there were. let $\frac{p}{q}$ be such a number where $p$ and $q$ have no common factors. Then  $(\frac{p}{q})^2=2$, 
    hence we have that $p^2=2q^2$, which makes  $p^2$ even, hence  $p$ is even. Then  $p=2k$ for some  $k \in \N$. Then 
    $2q^2=4k^2$, hence  $q^2=2k^2$ which is even, therefore  $q^2$ and consequently  $q$ is even; contradicting the fact that 
    $p$ and $q$ have no common factors. Therefore no such rational number exist.
\end{proof}
