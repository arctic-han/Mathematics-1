%----------------------------------------------------------------------------------------
%	SECTION 1.1
%----------------------------------------------------------------------------------------

\section{Convergent Sequences}

\begin{definition}
    A sequence $\{x_n\}$ in a metric space  $X$ is said to \textbf{converge} if there is 
    a point $x \in X$ such that for every  $\epsilon>0$, there is an  $ N \in \Z^+$ such that 
    $d(x_n,x)<\epsilon$ whenever $n \geq N$. We say $\{x_n\}$ \textbf{converges} to $x$, and we 
    call  $x$ the \textbf{limit} of  $\{x_n\}$ as  $n$ approaches  $\infty$. We write
    $x_n \rightarrow x$ as $n \rightarrow \infty$, and  $\lim_{n \rightarrow \infty}{x_n}=x$  (or $\lim{x_n}=x$).
    If $\{x_n\}$ does not converge, we say the  $\{x_n\}$ \textbf{diverges}, or \textbf{is divergent}.
\end{definition}

\begin{example}
    Consider the following sequences in $\C$.
        \begin{enumerate}[label=(\arabic*)]
            \item $\{\frac{1}{n}\}$ is bounded, and $\lim_{n \rightarrow \infty}{ \frac{1}{n}}=0$.

            \item The sequence $\{n^2\}$ us unbounded and diverges.

            \item $1+\frac{(-1)^n}{n} \rightarrow 1$ as $n \rightarrow \infty$, and  $\{1+\frac{(-1)^n}{n}\}$ is bounded.

            \item $\{i^n\}$ is bounded and divergent.

            \item  $\{1\}$ is bounded and converges to  $1$.
        \end{enumerate}
\end{example}

\begin{theorem}\label{3.1.1}
    Let $\{x_n\}$ be a sequence in a metric space, then:
        \begin{enumerate}[label=(\arabic*)]
            \item $\{x_n\}$ converges to  $x \in X$ if and only if every every neighborhood 
                of $x$ contains  $x_n$ for all but finitely many  $n$.

            \item If  $\{x_n\}$ converges to  $x$, and  $x'$, then $x=x'$.

            \item If  $\{x_n\}$ converges, then  $x_n$ is bounded.

            \item If  $E \subseteq X$, and  $x$ is a limit point of  $E$, then there 
                is a sequence  in  $E$ that converges to  $x$.
        \end{enumerate}
\end{theorem}


\begin{theorem}[The Sandwhich Theorem]\label{}

    Consider real valued sequences $\{x_n\}$,  $\{y_n\}$, and  $\{w_n\}$. Suppose that $\lim{x_n}=\lim{y_n}=a$ and that there 
    is an $N_0 \in \N$ sucht hat  $x_n \leq w_n \leq y_n$ for all  $n \geq N_0$. Then  $\lim_{n \rightarrow \infty}{w_n}=a$. 
\end{theorem}
\begin{proof}
    Let $\epsilon>0$ and let  $\{x_n\}$ and  $\{y_n\}$ both converge to  $a$. Then by definition there are  $N_1,N_2 \in \N$ 
    such that $|x_n-a|<\epsilon$ and  $|y_n-a|<\epsilon$ for  $n \geq N_1,n_2$. Now choose $N=\max{N_0,N_1,N_2}$, if 
    $n \geq N$, we have  $-\epsilon<x_n-a<\epsilon$, and we also have  $x_n-a<w_n-a<y_n-a$, thus we have that:
        \begin{equation*}
            -\epsilon<x_n-a<w_n-s<y_n-a<\epsilon
        \end{equation*}
    Thus we have that $|w_n-a|<\epsilon$.
\end{proof}

\begin{corollary}
    If $x_n \rightarrow \infty$ as  $n \rightarrow \infty$, and  $\{y_n\}$ is bounded, then $x_ny_n \rightarrow 0$ as 
    $n \rightarrow \infty$.
\end{corollary}
\begin{proof}
    We have that $\{y_n\}$ is bounded, hence, there is $M>0$ such that  $|y_n|<M$ for all  $n \in \N$. And since $\{x_n\}$ 
    converges to $0$ we have that for any $\epsilon$ there is an  $N \in \N$ such that for  $n \geq N$,  $|x_n-0|<\frac{\epsilon}{M}$.
    For $|x_ny_n-0|=|x_ny_n|<M|x_n|<M\frac{\epsilon}{M}=\epsilon$. Therefore, $x_ny_n \rightarrow 0$ as  $n \rightarrow \infty$.
\end{proof}

