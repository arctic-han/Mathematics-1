%----------------------------------------------------------------------------------------
%	SECTION 1.3
%----------------------------------------------------------------------------------------

\section{The Field of Real Numbers}

\begin{theorem}\label{1.3.1}
    There exists an ordered field $\R$ with the least upperbound property, such that 
     $\Q \subseteq \R$.
\end{theorem}

\begin{definition}
    We call the field $\R$ the \textbf{field of real numbers},and we call the elements 
    of $\R$ \textbf{real numbers}.
\end{definition}

\begin{definition}
    Let $S$ be an ordered field, and let  $E \subseteq S$. We say that  $E$ is \textbf{dense} 
    in $S$, if for all $r,s \in S$, with $r<s$, there is an $\alpha \in E$ such that 
    $r<\alpha<s$.
\end{definition}

\begin{theorem}[The Archimedean Principle]\label{1.3.2}
    If $x,y \in \R$, and  $x>0$, then there is an  $n \in \Z^+$ such that  $nx>y$.
\end{theorem}
\begin{proof}
    Let $A=\{nx: n \in \Z^+\}$, and suppose that  $nx \leq y$. Then  $y$ is an upperbound 
    of $A$, abd since  $A$ is nonempty,  $\alpha=\sup{A} \in \R$, since $x>0$, we have 
     $\alpha-x<\alpha$, so  $\alpha-x$ is not an upperbound of  $A$. Hence  $\alpha-x<mx$ for some 
     $m \in \Z^+$. Then  $\alpha<(1-m)x \in A$, contradicting that  $\alpha$ is an upperbound 
     of  $A$.
\end{proof}

\begin{theorem}[The density of $\Q$ in  $\R$]\label{1.3.3}
    $\Q$ is dense in  $\R$.
\end{theorem}
\begin{proof}
    Let $x<y$ be realnumbers, then  $y-x>0$, so by the Archimedean principle, there is 
    an $n \in \Z^+$ fir which $n(y-x)>1$. By the Archimedean principle again, we have  
    $m_1,m_2 \in \Z^+$ for which $m_1>nx$ and $m_2>-nx$, thus $-m_2<nx<m_1$, and we also 
    have that there is an $m \in \Z^+$ for which  $-m_2<m<m_1$, and $m-1 \leq nx<m$. Thus 
    combining inequalities, we get  $nx<m<ny$, thus  $x<\frac{m}{n}<y$.
\end{proof}

\begin{theorem}[The existence of $n^th$ roots of positive reals]\label{1.3.4}
    For every real number $X>0$, and for every  $n \in \Z^+$, there is one, and only one 
    positive real number  $y$ for which  $y^n=x$.
\end{theorem}
\begin{proof}
    Let $y>0$ be a real number; then $y^n>0$, so there is atmost one such  $y$ for which 
    $y^n=x$. Now let  $E=\{t:\R: t^n<x\}$, choosing  $t=\frac{x}{1+x}$, we see that 
    $0 \leq t<1$, hence  $t^n<t<x$, so $E$ is nonempty. Now if $1+x<t$, then $t^n \geq x$, so 
    $t \notin E$, and  $E$ has $1+x$ as an upperbound. Therefore, $\alpha=\sup{E} \in \R$ exists.

    Now suppose that  $y^n<x$, choose  $0 \leq h<1$ such that  $h<\frac{x-y^n}{n(y+1)^{n-1}}$, 
    then $(y+h)^n-y^n<hn(y+h)^{n-1}<hn(y+1)^{}n-1<x-y^n$, thus  $(y+h)^n<x$, so  $y+h \in E$, 
    contraditing that  $y$ is an upperbound. On the other hand, if  $y^n>x$, choosing 
    $k=\frac{y^n-x}{ny^{n-1}}$, then $0 \leq k<y$, and letting  $t \geq y-k$, we get that 
    $y^n-t^n \leq y^n+(y-k)^n<kny_{n-1}=y^n-x^n$, so $t^n \geq x$, making  $y-k$ an uppeerbound 
    of  $E$, which contradicts  $y=\sup{E}$.
\end{proof}
\begin{remark} 
    We denote $y$ as  $\sqrt[n]{x}$, or as  $x^{\frac{1}{n}}$.
\end{remark}

\begin{corollary}
    If $a,b \in \R$, with  $a,b>0$, and  $n \in \Z^+$, then  $\sqrt[n]{ab}=\sqrt[n]{a}\sqrt[n]{b}$.
\end{corollary}
\begin{proof}
    Let $\alpha=\sqrt[n]{a}$, and  $\beta=\sqrt[n]{b}$. Then  $\alpha^n=a$, and  $\beta^n=b$, so 
    $ab=\alpha^n\beta^n=(l\alpha\beta)^n$, we are done.
\end{proof}

\begin{definition}
    We define the \textbf{extended real number system} to be the field $\R$, together with symbols 
    $\infty$, and  $-\infty$, called  \textbf{positive infinity} and \textbf{negative infinity}, such that 
    $-\infty<x<\infty$ for all $x \in \R$. We call elements of the extended real numbers \textbf{infinite}, 
    and every other element not in the extended real numbers \textbf{finite}.
\end{definition}

\begin{lemma}\label{1.3.5}
    $\infty$ is an upperbound for every subset  $E$, of  $\R$, and  $-\infty$ is a lowerbound 
    for every subset  $E$ of  $\R$. Moreover, if $E$ is not bounded above, then  $\sup{E}=\infty$, 
    and if  $E$ is not bounded below, then  $\inf{E}=-\infty$.
\end{lemma}

\begin{remark}
    We make the following assumptions for extended real numbers:
        \begin{enumerate}[label=(\arabic*)]
            \item If $x \in \R$, then  $x+\infty=\infty$, $x-\infty=-\infty$, and  $ \frac{x}{\infty}=\frac{x}{-\infty}=0$.

            \item If $x>0$, then  $x(\infty)=\infty$ and  $x(-\infty)=-\infty$.

            \item If  $x<0$, then  $x(\infty)=-\infty$ and  $x(-\infty)=\infty$.
        \end{enumerate}
\end{remark}
