%----------------------------------------------------------------------------------------
%	SECTION 1.1
%----------------------------------------------------------------------------------------

\section{Ordered Sets}

\begin{definition}
    Let $S$ be any set. An \textbf{order} on $S$ is a relation  $<$ such that:
        \begin{enumerate}[label=(\arabic*)]
            \item For $x,y \in S$, one and only one of the following hold:
                \begin{align*}
                    x<y && x=y && y<x \\		
                \end{align*}
            We call this property the \textbf{trichotomy law}

            \item $<$ is transitive over $S$.
        \end{enumerate}
    We denote the relations $>$ and $\leq$ to mean $x>y$ if and only if $y<x$, 
    and $x \leq y$ if and only if $x<y$, or $x=y$. We call $S$ together with  $<$ 
    an \textbf{ordered set}.
\end{definition}

\begin{example}
    Define $<$  on $\Q$ such that for $r,s \in \Q$, $r<s$ implies $<0s-r$.		
\end{example}

\begin{definition}
    Let $S$ be an ordered set, and let  $E \subseteq S$. We say that  $E$ is \textbf{bounded above} 
    there is some  $\beta \in S$ for which  $x \leq \beta$, for all $x \in E$. We say that $E$ 
    is  \textbf{bounded below} if  $\beta \leq x$, for call  $x \in E$. We say an $\alpha \in S$ is a 
    \textbf{least upperbound} of  $E$, if  $\alpha$ is an upperbound of  $E$, and for all 
    other upperbounds,  $\gamma$, of  $E$,  $\alpha \leq \gamma$. Likewise,  $\alpha$ is a \textbf{greatest 
    lowerbound}  of $E$ if  $\alpha$ is a lowerbound of  $E$, and for all other lowerbounds  
    $\gamma$ of  $E$,  $\gamma \leq \alpha$. We denote the least upperbound, and greatest lowerbound 
    by  $\sup{E}$ and  $\inf{E}$, respectively.
\end{definition}

\begin{lemma}\label{1.1.1}
    Let $S$ be an ordered set, and let $E \subseteq S$. Then  $E$ has (if they exist) a 
    unique least upperbound, and a unique greatest lowerbound.
\end{lemma}
\begin{proof}
    Let $\alpha, \beta \in S$ be least upperbounds of $E$. Then by definition, we have that 
    $\alpha \leq \beta$, and  $\beta \leq \alpha$; thus by the trichotomy law,  $\alpha=\beta$. The proof is the same 
    for greatest lowerbounds.
\end{proof}

\begin{example}
    \begin{enumerate}[label=(\arabic*)]
        \item Let $A=\{p \in \Q: p^2<2\}$, and $B=\{p \in \Q: p^2>2\}$. Clearly, we 
            have that every element of $B$ is an upperbound of $A$, and every element of 
             $A$ is a lowerbound of  $B$. Now take  $p \in \Q$ a positive rational, and 
             take  $q \in \Q$ such that $q=p-\frac{p^2-2}{p+2}$. Then $q^2-2=\frac{2(p^2-2)}{(p+2)^2}$. 
             Now if $p \in A$, then  $p^2-2<0$, which implies that  $p<q$, and $q^2<2$; thus 
             $A$ has no largest element; similarly, if  $p \in B$, then  $p^2-2>0$, which 
             implies that  $q<p$ and  $q^2>2$, which shows that  $B$ has no least element. Thus 
             $\sup{A}$ and  $\inf{B}$ do not exist in $\Q$. 

         \item If  $\alpha = \sup{E} \in S$, it may or may not be that  $\alpha \in E$. Take 
             $E_1=\{r \in \Q: r<0\}$, and  $E_2=\{r \in \Q: r \leq 0\}$. Then $\sup{E_1}=
             \sup{E_2}=0$, but $0 \notin E_1$, where as $0 \in E_2$

         \item COnsider the set $ \frac{1}{\Z^+}=\{\frac{1}{n}: n \in \Z^+\}$. By the 
             well ordering principle, $1$ is the least element, and is also an upperbound 
             of all  $ \frac{1}{n}$ for $n>1$. Now also notice that as $n$ gets arbitrarily 
             large, then  $ \frac{1}{n}$ gets arbitratirly small; that is to say $\frac{1}{n}$ 
             ``tends'' to $0$, so  $\sup{\frac{1}{\Z^+}}=1 \in \frac{1}{\Z^+}$, and 
             $\inf{\frac{1}{\Z^+}}=0 \notin \frac{1}{\Z^+}$.
    \end{enumerate}		
\end{example} 

\begin{definition}
    We say an ordered set $S$ has the \textbf{least upperbound property}, if whenever 
    $E \subseteq S$, nonempty, and bounded above, then  $\sup{E} \in E$; likewise, $S$ has 
    the \textbf{greatest lowerbound property} if whenever $E$ is nonempty, bounded below 
    then $\inf{E} \in E$
\end{definition}

\begin{example}
    The set of all rationals $\Q$ does not have the least upperbound property, nor the 
    greatest lowerbound property, take $A$,  $B$ as in the previous example. Letting  
    $E=\{1, \frac{1}{2}, \frac{1}{4}\} \subseteq \frac{1}{\Z^+}$, we see that $\frac{1}{\Z^+}$ 
    satisfies both properties, with $\sup{E}=1$, and  $\inf{E}=\frac{1}{4}$.
\end{example} 

\begin{theorem}\label{1.1.2}
    If $S$ is an ordered set with the least upperbound property, then $S$ also inherits 
    the greatest lowerbound property.
\end{theorem}
\begin{proof}
    Let $B \subseteq S$, and let  $L \subseteq S$ be the set of all lowerbounds of $B$. Then we have 
    for any $y \in L$, $x \in B$, $y \leq x$. So every element of  $B$ is an upperbound of  $L$, and 
    $L$ is nonempty, hence $\alpha=\sup{L} \in S$ exists. Now if  $\gamma \leq \alpha$, then 
     $\gamma$ is not an upperbound of  $L$, hence  $\gamma \notin B$; thus  $\alpha \leq x$ for all 
      $x \in B$, so  $\alpha \in L$, and by definition of the greatest lowerbound, we get 
      $\alpha=\inf{B}$.
\end{proof}
