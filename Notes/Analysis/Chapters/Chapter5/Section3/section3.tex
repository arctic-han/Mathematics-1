%----------------------------------------------------------------------------------------
%	SECTION 1.1
%----------------------------------------------------------------------------------------

\section{Continuity and Compactness.}

\begin{definition}
    A mappinf $f:E \rightarrow \R^k$ is said to be \textbf{bounded} if there is a real number $M>0$ 
    such that  $||f|| \leq M$ for all  $x \in E$.
\end{definition}

\begin{theorem}\label{5.3.1}
    Let $f$ be a cn=ontiuous mapping of a compact metric space  $X$ into a metric space  
    $Y$. Then  $f(X)$ is compact in  $Y$.
\end{theorem}
\begin{proof}
    Let $\{V_{\alpha}\}$ be an open cover of  $f(X)$, since $f$ is contiuous, then  $f^{-1}(V_{\alpha})$ 
    is open in  $X$, and since  $X$ is compact,  $X \subseteq \bigcup_{i=1}^{n}{V_{\alpha_i}}$, and 
    $f(f^{-1}(E)) \subseteq E$, we have that  $f(X) \subseteq \bigcup_{i=1}6{n}{V_{\alpha_i}}$.
\end{proof}

\begin{theorem}\label{5.3.2}
    If $f:X \rightarrow \R^k$ is continuous, where  $X$ is a compact metric space, then  $f(X)$ 
    is closed and bounded; in particular, $f$ is bounded.
\end{theorem}
\begin{proof}
    From theorem \ref{5.3.1}, we have that $f(X)$ is compact in  $\R^k$, therefore, it is 
    closed and bounded.
\end{proof}

\begin{theorem}[The Extreme Value Theorem]\label{5.3.3}
    Suppose $f$ is a continuous, realvalued function on a metric space $X$, and that  \
    $M=\sup{f}$, and  $m=\inf{f}$. Then there exist points  $p,q \in X$ such that  $f(p)=M$ 
    and  $f(q)=m$.
\end{theorem}
\begin{proof}
    By theorem \ref{5.3.2}, $f(X)$ is closed and bounded, thus  $M,m \in f(X)$.
\end{proof}

\begin{theorem}\label{5.3.4}
    Suppose $f$ is a continuous 1-1 mapping of a compact metric space  $X$ onto a metric 
    space  $Y$. Then the inverse mapping  $f^{-1}:Y \rightarrow X$ is a Continuous mapping 
    of  $Y$ onto  $X$.
\end{theorem}
\begin{proof}
    By theorem \ref{5.2.3}, it suffices to show that  $f(V)$ is open in  $Y$ whenever  $V$ is 
    open in  $X$. We have that  $\com{X}{V}$ is closed in  $X$, and compact, thus  $f(\com{X}{V})$ 
    is closed and compact in  $Y$, thus $ f(V)=\com{Y}{f(\com{X}{V})}$ is open in $Y$.
\end{proof}

\begin{definition}
    Let $f$ be a mapping of a metric space  $X$ into a metric space $Y$. We say that  $f$ is 
    \textbf{uniformly continuous} on $X$ if for every  $\epsilon>0$, there is a  $\delta>0$ such that 
    $d_Y(f(q),f(p))<\epsilon$, for all  $p,q \in X$ for which  $d_X(p,q)<\delta$.
\end{definition}

\begin{lemma}\label{5.3.5}
    If $f$ is uniformly continuous, then  $f$ is contiuous.
\end{lemma}

\begin{theorem}\label{5.3.6}
    Let $f$ be a continuous mapping of a compact metric space $X$ into a metric space $Y$. 
    Then  $f$ is uniformly continuous on  $X$
\end{theorem}
\begin{proof}
    Let $\epsilon>0$, by the continuity of  $f$, we can associate for each  $p \in X$ a number 
    $\phi(p)>0$ such that  for $q \in X$,  $d_X(p,q)<\phi(p)$ implies $d_Y(f(p),f(q))<\frac{1}{2}\phi(p)$. 
    Now let $J(p)=\{q \in X:d_X(p,q)<\pghi(p)\}$. Clearly,  $p \in J(p)$, so  $J(p)$ is 
    an open cover of  $X$, and since  $X$ is compact, there are  $p_1, \dots, p_n$ for which 
    $X \subseteq \bigcup_{i=1}^{n}{J(p_i)}$, then take  $\delta=\min\{\phi(p_1), \dots, \phi(p_n)\}$; we have 
    $\delta>0$. Now let  $p,q \in X$ such that  $d_X(p,q)<\delta$. Then  there is an $m \in \Z^+$ 
    with  $1 \leq m \leq n$ such that  $p \in J(p_m)$, thus  $d_X(p,q)<\frac{1}{2}\phi(p_m)$, 
    by the triangle inequality, we get $d_(q,p_m) \leq d_X(q,p)+d_X(p,p_m)<\delta+\frac{1}{2}\phi(p_m)=\phi(p_m)$, 
    for $1 \leq m \leq n$. Therefore,  $d_Y(f(p),f(q)) \leq d_Y(f(p),f(p_m))+d_Y(f(p_m),f(q))<\epsilon$. 
    Thus,  $f$ is uniformly contiuous.
\end{proof}

\begin{remark}
    What this theorem says, is that in any compact metric space, continuity and uniform 
    continuity are equivalent.
\end{remark}

\begin{theorem}\label{5.3.7}
    Let $E \subseteq \R$ be noncompact, then:
        \begin{enumerate}[label=(\arabic*)]
            \item There exists a continuous function on $E$ which is not bounded.

            \item There is a bounded, continuous function on $E$ which has no maximum.

            \item If  $E$ is bounded, there exists a continuous function on  $E$ that is 
                not uniformly continuous.
        \end{enumerate}
\end{theorem}
\begin{proof}
    Suppose first that $E$ is bounded. Then there is a limit point  $x_0 \notin E$ of $E$. 
    Consider the function
        \begin{equation*}
            f(x)=\frac{1}{x-x_0} \text{ for all } x \in E
        \end{equation*}
    Then $f$ is continuous on  $E$, but not bounded. Then let  $\epsilon>0$ and  $\delta>0$, and 
    choose  $x \in E$ such that  $|x-x_0|<\delta$, then taking $t$ arbitrarily close to  $x_0$, 
    we can get $|f(x)-f(t)| \geq \epsilon$, even though  $|x-t|<\delta$. Thus  $f$ is not 
    uniformly continuous.

    Now choose 
        \begin{equation*}
            g(x)=\frac{1}{1+(x-x_0)^2} \text{ for all } x \in E
        \end{equation*}
    g is continuous, and bounded on $E$ (0<g<1), then $\sup{g}=1$, and since  $g(x)<1$ for all 
    $x$, we see that  $g$ attains no maximum.

    Lastly, suppose that $E$ is unbounded, then the functions  $f(x)=x$ and  $h(x)=\frac{x^2}{1+x^2}$ 
    for all $x \in E$ establish  $(1)$ and  $(2)$.
\end{proof}

\begin{example}
    Let $f$ be the mapping of the interval  $[0,2\pi)$ onto the unit circle. That is 
    $f(t)=(\cos{t},\sing{t})$ for  $0 \leq t<2\pi$. Then  $f$ is a continuous 1-1 mapping of 
    $[0,2\pi)$ onto the unit circle, however, the inverse mapping,  $f^{-1}$ fails to be 
    continuous at  the point $f(0)=(1,0)$.
\end{example} 
