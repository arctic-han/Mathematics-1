%----------------------------------------------------------------------------------------
%	SECTION 1.2
%----------------------------------------------------------------------------------------

\section{Subrings and Subfields of $\C$.}
\label{section2}

We restrict our notion of ``subrings'' and ``subfields'' to those concerning 
$\R$ and  $\C$.

\begin{definition}
    A \textbf{subring} of $\C$ is a nonempty set  $R \subseteq \C$ such that 
    $1 \in R$, and if  $x,y \in R$ then  $x+y,-x \in R$ and  $xy \in R$.
\end{definition}

\begin{definition}
    A \textbf{subfield} of $\C$ is a subring  $K \subseteq \C$ such that if $x \in K$ 
    then $x^{-1} \in K$.
\end{definition}

Since, we are talking about subrings and subfields in the sense of  $\R$ and  $\C$, 
then we denote  $x^{-1}$ to be $\frac{1}{x}$.

\begin{example}
    \begin{enumerate}[label=(\arabic*)]
        \item The set $\Z[i] \subseteq \C$ of all pairs of integers $(a,b)$ of 
            the form  $a+ib$ is a subring of  $\C$ but not a subfield. We call 
            this set the \textbf{Gaussian integers}.

        \item The set $\Q[i] \subseteq \C$ of all pairs of rational numbers $(p,q)$ 
            of the form  $a+ib$ forms not only a subring of  $\C$, but also a 
            subfield.

        \item The set  $P[\pi]$ of all polynomials in  $\pi$ with rational 
            coefficients is a subring of  $\C$, but not a subfield.

        \item The set  $\Q(\pi)$ of all rational expressions in  $\pi$, $ 
            \frac{p(\pi)}{q(\pi)}$ (with $q(\pi) \neq 0$) with rational coeffients 
            is a subfield of $\C$.

        \item The set  $2\Z$ of all even integers is not a subring of  $\C$.

        \item The set  $\Q[\sqrt[3]{2}]$ of all pairs of rationla numbers $(a,b)$ 
            of the form  $a+b\sqrt[3]{2}$ does not form a subring of $\C$ since 
            it is not closed under  $\cdot$. It is closed however under  $+$.
    \end{enumerate}
\end{example} 

\begin{definition}
    Let $K$ and  $L$ be subfields of  $\C$. An \textbf{isomorphism} between  $K$ 
    and $L$ is a $1-1$ mapping $\phi:K \rightarrow L$ from $K$ onto  $L$ such 
    that $\phi(x+y)=\phi(x)+\phi(y)$ and $\phi(xy)=\phi(x)\phi(y)$ for all $x,y 
    \in K$.
\end{definition}

\begin{proposition}\label{proposition1.2.1}
    If $\phi:K \rightarrow L$ is an isomorphism, then:
        \begin{enumerate}[label=(\arabic*)]
            \item $\phi(0)=0$.

            \item  $\phi(1)=1$.

            \item  $\phi(-x)=-\phi(x)$. 

            \item $\phi(x^{-1})=\phi(x)^{-1}$
        \end{enumerate}
\end{proposition}
\begin{proof}
    \begin{enumerate}[label=(\arabic*)]
        \item We have $0x=0$ for all  $x \in K$, so  $\phi(0x)=\phi(0)\phi(x)=
            \phi(0)$. Let  $\phi^{-1}(0)=x$, so  $\phi(0)\phi^{-1}(0)=
            \phi\phi^{-1}(0)=0$.

        \item This is esentially the same proof as $(1)$, but with  $1$ instead 
            of  $0$.

        \item We have that  $x+(-x)=0$, so $\phi(x+(-x))=\phi(x)+\phi(-x)=\phi(0)
            =0$, hence we get that $\phi(-x)=-\phi(x)$.

        \item This proof is analogous to that of $(3)$.
    \end{enumerate}		
\end{proof}

If $\phi:K \rightarrow L$ is  $1-1$, but not necesarrily onto, then we call it a 
\textbf{monomorphism}. If $L=K$, then we call  $\phi$ an \textbf{automorphism}.

 \begin{definition}
    A \textbf{primitive n-th root of unity} is an $n$-th root of $1$ that is not 
    an $m$-th root of  $1$ for any proper divisor  $m$ of  $n$.
\end{definition}

\begin{example}
    We have that $i$ is a  $4$th root of unity, as  $i^4=(i^2)^2=(-1)^2=1$. The 
    number $\zeta_n=e^{2i\pi/n}$ is also an $n$th root of unity.
\end{example} 

\begin{proposition}\label{proposition1.2.2}
    Let $\zet_n=e^{2i\pi/n}$, then $\zeta_n^k$ is a primitive  $n$th root of 
    unity if and only if  $\gcd(k,n) = 1$.
\end{proposition}
\begin{proof}
    We prove the contrapositive of this proposition. Suppose that $\zeta_n^k$ is 
    not a primitive  $n$th root of unity, then  $(\zeta_n^k)^m=1$ for some  $n$ 
    such that $n=mr$. Then  $\zeta_n^{km}=1=\zeta_n ^n$, hence $n=mr|kr$, 
    therefore  $r|k$, and since  $r|n$, then  $\gcd(k,n) \geq r>1$  (by definition 
    of the greatest common divisor).

    Now suppose that  $\gcd(k,n)=r>1$. Then  $n=mr$ for some  $m \in \Z$, and  
    $n=mr|km$ for some  $k$. Thus we get that  $(\zeta_n^k)^m=\zeta_n^{km}=1$, 
    therefore,  $\zeta_n^k$ is not a primitive  $n$th root of unity.
\end{proof}

\begin{example}
    \begin{enumerate}[label=(\arabic*)]
        \item \\textbf{complex conjugation} defined as the mapping $(x,y) 
            \rightarrow (x,-y)$ (that is, $x+iy \rightarrow x-iy$) is an automorphism 
            from $\C$ onto  $\C$.
        \item Let  $\Q[\sqrt{2}]$ be the set of all pairs of $(p,q)$ rational 
            numbers of the form $p+q\sqrt{2}$. Then  $\Q[\sqrt{2}]$ is a subfield 
            of  $\C$. The map  $p+q\sqrt{2} \rightarrow p-q\sqrt{2}$ is an 
            automorphism from  $\Q[\sqrt{2}]$ onto itself.

        \item Lel  $\alpha=\sqrt[3]{2}$ and let  $\omega=\frac{1}{2}+
            i\frac{\sqrt{3}}{2}$ be a primitive cube root of unity in $\C$. Then 
            the set of all $\Q[\alpha]$ triples of rational numbers  $(p,q,r)$ of 
            the form $p+q\alpha+r\alpha^2$ is a subfield of  $\C$. The map 
            $p+q\alpha+r\alpha^2 \rightarrow p+q\omega\alpha+r\omega\alpha^2$ is an monomorphism onto its image, 
            but not an automorphism.
    \end{enumerate}		
\end{example} 
