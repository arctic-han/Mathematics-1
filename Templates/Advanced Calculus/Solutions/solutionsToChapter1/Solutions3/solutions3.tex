%----------------------------------------------------------------------------------------
%	SECTION X.X
%----------------------------------------------------------------------------------------

\section{The Axiom of Completeness.}
\begin{enumerate}[label=(\arabic*)]
    \item[2] Suppose $E \subseteq \Z$ and  $E \neq \emtyset$ and that  $E$ is bounded. Then  $E$ is bounded above, so by 
        the axiom of completeness, $\sup{E}$ exists. Now by the approximation property, fo  $\epsilon>0$ and  $a \in \Z$ 
        $\sup{E}-\epsilon \leq a \leq \sup{E}$ ; making $\epsilon$ negligable enough we see that  $\sup{E} \in \Z$. Since$E$ 
        has infinitely many upperbounds, choose  $M \in \Z$  to be the least uppeerbound of  $E$ not in  $E$ and  
        $M \neq \sup{E}$, then $n \leq \sup{E} \leq M$, which makes  $\sup{E} \in E$.

        Again, by the aboive agrgument, using the approximation property for greatest lower bounds  (which we prove later) 
        and lwtting $m \in \Z$ the largest lowerbound not in  $E$ with  $m \neq \inf{E}$, we see that  $\inf{E} \in E$.

    \item[3] Let  $a,b \in \R$ with  $a<b.$, then notice that  $a-\sqrt{2}<b-\sqrt{2}$. By the density of  $\Q$ in  $\R$, there 
        is a $\frac{p}{q}$ such that $a-\sqrt{2}<\frac{p}{q}<b-\sqrt{2}$. Then $a<\frac{p}{q}+\sqrt{2}<b. Now $\frac{p}{q} \in \Q$, 
        but $\sqrt{2} \notin \Q$; thus $\frac{p}{q}+\sqrt{2} \notin \Q$, therefore $\Q^*$ is dense in  $\R$.

    \item[4] Let  $a \in \R$ and  $n \in \N \subseteq \R$, Now either  $a<n$,  $n<a$ or $a=n$. Suppose that  $a<n$, then 
         $\frac{}{}rac{1}{n}<\frac{1}{a}<a$, so $\frac{1}{n}<a$. By the density of $\Q$ in  $\R$, there is an  $r_n \in \Q$ such that 
         $\frac{1}{n}<r_n<a$. Now $r_n-\frac{1}{n}<a-\frac{1}{n}<a$, so $r_n-\frac{1}{n}<a$, it remains to prove that 
         $a<r_n+\frac{1}{n}$. We have that $nr_n<na$, then  $$na-1<nr_n$, then  $na<nr_n+1$, thus  $a<r_n+\frac{1}{n}$. There 
         fore we have $r_n-\frac{1}{n}<a<r_n+\frac{1}{n}$, so $|a-r_n|<\frac{1}{n}$. The proof for the case of $n<a$ is analogous;
         and if  $a=n$, then the case reduces to the one we proved, with  $\frac{1}{a}<a$.
         
     \item[5(a)] \begin{theorem}[Approximation Property for Greatest Lowerbounds]
                If $E$ has a least upperbound, and $\epsilon>0$ is a positive number, then there is an element $a \in E$ such 
                that $\inf{E} < a \leq \inf{E}+\epsilon$.

            \end{theorem}
            \begin{proof}
                Suppose for some $\epsion_0>0$ that there is no element of  $E$ lying betweenw  $\inf{E}$ and $\inf{E}+\epsion_0$. 
                Now $\inf{E}+\epsion_0 \leq a$ for all  $a \in E$, so  $\inf{E}+\epsion_0$ is a lower bound, thus 
                $\inf{E}+\epsion_0 \leq \inf{E}$, implying that $\epsion_0 \leq 0$, a contradiction. 
            \end{proof}

        \item[6]
            \begin{enumerate}
                \item Suppose that $E$ has a lower bound  $m$. Then for some  $m_0<m$, $m_0$ is also a lower bound for $E$. 
                    Now let  $m_0,m_1$ be greatest lowerbounds of $E$, then  $m \leq m_0$ for all other lowerbounds  $m$, likewise 
                     $m \leq m_1$. Then we get that $m_1$ is a lowerbound, so $m_1 \leq m_0$, likewisem $m_0 \leq m_1$, therefore 
                     $m_0=m_1$ and the greatest lowerbound is unqiue.

                \item Let $E \subseteq \R$ be nonempty and bounded below, then  $-E \neq \emptyset$ and bounded above, therefore 
                    by the axiom of completeness,  $\sup{-E}$ exists, and by theorem  $1.3.4$,  $\sup{-E}=-\inf{E}$, which 
                    implies that  $\inf{E}$ exists (and is finite).
            \end{enumerate}

\end{enumerate}
