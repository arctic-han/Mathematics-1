%% LyX 2.2.3 created this file.  For more info, see http://www.lyx.org/.
%% Do not edit unless you really know what you are doing.
\documentclass[12pt,oneside,english]{amsbook}
\usepackage[T1]{fontenc}
\usepackage[latin9]{inputenc}
\usepackage{amsbsy}
\usepackage{amstext}
\usepackage{amsthm}
\usepackage{amssymb}
\usepackage{graphicx}
\usepackage{setspace}
\doublespacing

\makeatletter
%%%%%%%%%%%%%%%%%%%%%%%%%%%%%% Textclass specific LaTeX commands.
\numberwithin{section}{chapter}
\numberwithin{equation}{section}
\theoremstyle{plain}
\newtheorem{thm}{\protect\theoremname}
  \theoremstyle{definition}
  \newtheorem{example}[thm]{\protect\examplename}
  \theoremstyle{remark}
  \newtheorem{rem}[thm]{\protect\remarkname}
  \theoremstyle{plain}
  \newtheorem{lem}[thm]{\protect\lemmaname}
  \theoremstyle{plain}
  \newtheorem{prop}[thm]{\protect\propositionname}
  \theoremstyle{definition}
  \newtheorem{defn}[thm]{\protect\definitionname}
  \theoremstyle{plain}
  \newtheorem{cor}[thm]{\protect\corollaryname}

\makeatother

\usepackage{babel}
  \providecommand{\corollaryname}{Corollary}
  \providecommand{\definitionname}{Definition}
  \providecommand{\examplename}{Example}
  \providecommand{\lemmaname}{Lemma}
  \providecommand{\propositionname}{Proposition}
  \providecommand{\remarkname}{Remark}
\providecommand{\theoremname}{Theorem}

\begin{document}

\chapter{Introduction}

The experimental and discrete mathematics play a very important role
in the development of this thesis. It's precisely in these two branches
of research that the family of symmetric functions defined over Galois
fields has reached a special interest. Moreover, it could be said
that in the last decades the family of symmetrical Boolean functions
has acquired a greater importance in the area of applicable mathematics.
Thus, over time, the study of these functions have helped to obtain
great advances in applications, for example, today they are active
areas of research as the experimental mathematical, combinatorial
theory, coding theory, cryptography, the theory of error-correcting
codes, game theory and electrical engineering (see \cite{BCP,cai,cm1,cm3,cusickstanica,dalaimaitrasarkar,hell,stanicamaitra,stanicamaitraclark}). 

Below we present some of the main components of this thesis in a brief
conceptual framework taking as an example, the Boolean space. That
is, all the mathematical objects mentioned in this first part of the
introduction are defined in a similar way for other Galois fields
($\mathbb{F}_{q}$ with $q$ a power of a prime $p$). similar all
the results of the research. To begin with, one of the firsts mathematical
objects used and highlighted in this section is that of a Boolean
function. A Boolean function is defined as
\begin{equation}
F^{*}:\mathbb{F}_{2}^{n}\rightarrow\mathbb{F}_{2}.
\end{equation}
For the purpose of this investigation, we frequently use the polynomial
representation of $F^{*}$ known as algebraic normal form (ANF) of
$F^{*}$ and we identify it as a function $F$ in $n$ variables over
$\mathbb{F}_{2}$ \cite{canteaut,licusick1}. In addition it should
be noted that, within the family of Boolean functions, there are functions
that fulfill an important property, that of being symmetric. It is
said that a Boolean function $F$ is symmetric if it is verified that
\begin{equation}
F\left(x_{1},\cdots,x_{n}\right)=F\left(x_{\sigma\left(1\right)},\cdots,x_{\sigma\left(n\right)}\right),
\end{equation}
where $\sigma$ is any permutation of the symmetric group $S_{n}$.
In other words, for simplicity it is said that $F$ is symmetric if
the subindices of $\left(x_{1},\cdots,x_{n}\right)$ can be permuted
so that the resulting function is invariant. Within the family of
the symmetric Boolean functions, two important functions that will
be studied in this investigation are highlighted: the elementary symmetric
Boolean polynomial and the rotation symmetric Boolean function. The
symmetric Boolean function can be associated with an exponential sum
and define as a second main mathematical object. The exponential sum
of symmetric Boolean function is defined as
\begin{equation}
S\left(F\right)=\sum_{X\in\mathbb{F}_{2}^{n}}\left(-1\right)^{F\left(X\right)}.
\end{equation}
To continue complementing the mathematical objects used in this thesis,
especially in Chapter $2$, knowledge of the properties of homogeneous
linear recurrence sequences with integer coefficients is required
(see next section). The sequences of exponential sums, such as the
sequences of exponential sums of elementary symmetric functions and
rotation symmetric functions are well studied throughout this workt.
In particular, for research on the sequences of exponential sums of
symmetric Boolean functions, it is very important to highlight the
following relation
\begin{equation}
wt(F)=\frac{2^{n}-S(F)}{2},\label{eq:4}
\end{equation}
where $\text{wt}(F)$ is the weight of Hamming ($\text{wt}(F)$ counts
the number of non-zero components of the $W_{F}=\left(F\left(X_{1}\right),\ldots,F\left(X_{2^{n}}\right)\right)$)
\cite{cusick4,cusickArXiv,cusickjohns,cusickstanica}. More precisely,
the equation (\ref{eq:4}) postulates that, any sequence of exponential
sums of symmetric Boolean functions satisfying a homogeneous linear
recurrence with integer coefficients, would lead to the respective
Hamming weights sequence of these functions, also would satisfy one
of these homogeneous linear recurrence (see Chapter $2$). In other
investigations where equation (\ref{eq:4}) \cite{cm1,cusickArXiv}
is studied, known that this relationship involves two important properties
over Boolean functions, such properties are the weight of Hamming
and the exponential sum. In the area of cryptography, in particular,
the criterion of the balanced of a Boolean function is investigated;
in this case, equation (\ref{eq:4}) us allows to relate the balancing
of an elementary symmetric Boolean polynomial $\mathcal{F}$ ($W_{\mathcal{F}}$
has the same number of zero and one's components) through its exponential
sum \cite{cm3,cusick2}. For example, a Boolean function $F$ is balanced
if and only if $S\left(F\right)=0$.

An important aspect in this research and presented in Chapter $3$,
is the efficiency in the computation of the implementations of the
closed formulas established in the results of this chapter, that is,
in this thesis we present closed formulas of efficient calculations.
For example, the closed formula of exponential sums that generalize
the result of Cai, Green, and Thierauf \cite{cai}. From the definition
itself, several of the properties on the mathematical objects mentioned
previously would be quite expensive to verify them, for example, the
verification of the balanced of a Boolean function using the Hamming
weight of a symmetric Boolean function is not very efficient. Moreover,
the closed formulas of Chapter $3$ corroborate or support the verification
of the Stanica-Li-Cusick conjecture asymptotically for Galois fields
\cite{cm1}.

Already on Galois fields, we present an important relationship that
opens a new approach for future research. In Chapter $3$ we present
closed formulas that link the exponential sums of symmetric polynomials
over any Galois field with one related to the problem of the bisecting
binomial coefficients. A solution $(\delta_{0},\delta_{1},\cdots,\delta_{n})$
to the equation 
\begin{equation}
\sum_{j=0}^{n}\delta_{j}\binom{n}{j}=0,\,\,\,\delta_{j}\in\{-1,1\},
\end{equation}
is said to give a {\em bisection of the binomial coefficients}
$\binom{n}{j}$, $0\leq j\leq n.$ To end this part of the chapter,
we define next the exponential sum of a symmetric polynomial $P$
defined over $\mathbb{F}_{q}$ as
\begin{equation}
S_{\mathbb{F}_{q}}\left(P\right)=\sum_{x\in\mathbb{F}_{q}^{n}}\zeta_{p}^{Tr_{\mathbb{F}_{q}/\mathbb{F}_{p}}\left(P\left(x\right)\right)}
\end{equation}
where $\zeta_{p}=e^{\frac{2\pi i}{p}}$ and $Tr_{\mathbb{F}_{q}/\mathbb{F}_{p}}$
is the trace function defined from $\mathbb{F}_{q}$ to $\mathbb{F}_{p}$.
The symmetric polynomial $P$ and its trace (defined over Galois fields),
is in active research, for example, in the case of balanced of $P$
are directly related to the study of the trace function \cite{acgmr}.
In the following sections, we introduce a summary on the theme of
linear recurrences of symmetric functions, associated recursions of
special functions and closed formulas of symmetric polynomial over
Galois fields (Chapters $2$ and $3$ respectively).

\section{Sequence of linear recurrences}

A {\em sequence of linear recurrences\textit{ $s$}} is as infinite
sequence $s_{0},s_{1},\cdots$ of elements in $\mathbb{Z}\left[\zeta_{p}\right]$
where $\zeta_{p}=e^{\frac{2\pi i}{p}}$, having the following property:
there are constant $a_{0},a_{1},\cdots,a_{k-1}$ (con $a_{0}\neq0$)
such that, for all $n\geq0$,
\begin{equation}
s_{n+k}=a_{k-1}s_{n+\left(k-1\right)}+\cdots+a_{1}s_{n+1}+a_{0}s_{n}+a\label{eq:7}
\end{equation}
If initial values $s_{0},s_{1},\cdots,s_{k-1}$ of the sequence are
provided, the recurrence relation defines the rest of the seuqence
uniquely. Such a sequence es said to order $k$. Also we say that
the recurrrence (\ref{eq:7}) is homogeneous, if $a=0$ . For example,
the exponential sums of the Boolean function
\[
{\scriptstyle {\scriptstyle F_{n}({\bf X})=X_{1}X_{2}X_{4}+X_{2}X_{3}X_{5}+\cdots+X_{n-3}X_{n-2}X_{n}+X_{n-2}X_{n-1}X_{1}+X_{n-1}X_{n}X_{2}+X_{n}X_{1}X_{3}}}
\]
{\small{}satisfies homogeneous linear recurrence with integer coefficients
of order }$6$ , that is, let $X\in\mathbb{F}_{2}^{n}$, the sequence
$\left\{ S\left(F_{n}({\bf X})\right)\right\} _{n\geq4}$ satisfies
\[
a_{n}=2a_{n-1}+2a_{n-2}-4a_{n-3}+4a_{n-5}-8a_{n-6}
\]
where the first integers of recurrence are 
\[
8,20,16,56,32,144,144,352,512,832,1600,\cdots,
\]
also satisfies homogeneous linear recurrence its weight sequences
$\left\{ wt\left(F_{n}({\bf X})\right)\right\} $ (see more in the
Chapter $2$ ) . Other example, the next symmetric function of order
$3$
\[
e_{n,3}=X_{1}X_{2}X_{3}+X_{1}X_{2}X_{4}+\cdots+X_{n-3}X_{n-2}X_{n}
\]
also generates the sequence $\left\{ S_{\mathbb{F}_{4}}\left(e_{n,3}\right)\right\} _{n\geq3}$
{\small{}that satisfies homogeneous linear recurrence with integer
coefficients of order }$3$ and by theorem $\left(\ref{MoregeneralTHMSymmetric}\right)$
of Chapter $3$, you have the formula closed
\begin{eqnarray*}
S_{\mathbb{F}_{4}}(\boldsymbol{e}_{n,3}) & = & 4^{n-1}+3\cdot2^{n-1}-3\cdot2^{n-1}\cos\left(\frac{n\pi}{2}\right).
\end{eqnarray*}

The homogeneous sequence that satisfies$\left(\ref{eq:7}\right)$
can be associated with the following polynomial,
\[
P_{k}\left(X\right)=X^{k}-a_{k-1}X^{k-1}-\cdots-a_{1}X-a_{0}
\]
known as the {\em polynomial characteristic of $s$} or to the
linear recurrence sequence$\left\{ s_{n}\right\} _{n\in\mathbb{N}}$.
In practice, when all the roots $\lambda_{i}$ of the polynomial $P_{k}\left(X\right)$
are different, then the expression $\sum_{i=1}^{k}a_{i}\lambda_{i}^{n}$
satisfies linear recurrence sequence of the form $\left(\ref{eq:7}\right)$.
Moreover, if we denote $s_{n}$ by $\sum_{i=1}^{k}a_{i}\lambda_{i}^{n}$,
we can say that the sequence $s_{0},s_{1},\cdots$ satisfies a homogeneous
linear recurrence of the form $\left(\ref{eq:7}\right)$ such that
\[
P_{k}\left(X\right)=\prod_{i=1}^{k}\left(X-\lambda_{i}\right)
\]
 is its characterist polynomial. 
\begin{example}
If we apply a result of Cai, Green, and Thierauf , the theorem $3.2$
\cite{cai} or the theorem $3.1$ \cite{cm1}, the sequence $\left\{ S(\boldsymbol{e}_{n,2^{l}})\right\} $
satisfies a homogeneous linear recurrence whose characteristic polynomial
is
\begin{equation}
P_{2^{r}-1}\left(X\right)=\prod_{i=0}^{2^{r}-1}\left(X-\lambda_{i}\right)
\end{equation}
where $\lambda_{j}=1+\zeta_{j}$ and $\lambda_{0}=2^{n}$ are the
roots of the polynomial (all different). In other words, we can for
a fixed integer $l$, then for all $n\geq2^{l}$
\begin{equation}
S(\boldsymbol{e}_{n,2^{l}})=c_{0}(2^{l})2^{n}+\sum_{j=1}^{2^{l+1}-1}c_{j}(2^{l})(1+\zeta_{j})^{n},\label{eq:9}
\end{equation}
otras where $\zeta_{j}=e^{\frac{\pi i\,j}{2^{l}}},i=\sqrt{-1}$ and
\[
c_{j}(2^{l})=\frac{1}{2^{l+1}}\sum_{t=0}^{2^{l+1}-1}(-1)^{\binom{t}{2^{l}}}\zeta_{j}^{-t}.
\]
by the theorem (\ref{MoregeneralTHMSymmetric}) the sequence $\left\{ S(\boldsymbol{e}_{n,2^{l}})\right\} $
and its exponential sum $S(\boldsymbol{e}_{n,2^{l}})$ expressed as
a closed formula (see equation \ref{eq:9}) {\small{}satisfies homogeneous
linear recurrence with integer coefficients }{\small \par}
\end{example}


\section{Recursion associated to some special functions}

In the theorem $\left(\ref{MoregeneralTHMSymmetric}\right)$ of the
Chapter $2$, establishes that for $n\geq k$, the elementary symmetric
polynomial $e_{n,k}$ associated with the exponential sum $S\left(\boldsymbol{e}_{n,k}\right)$
generates a homogeneous recurrent sequence of linear origin and with
integer coefficients. Similarly by the theorem L of chapter 2 it is
established in the same way that other two important functions satisfy
homogeneous linear recurrence, we enumerate them, the first is the
function initially defined by Pieprzyk an Qu in \cite{piequ} called
the rotation symmetric Boolean function. Some examples of rotation
symmetric functions are given by
\begin{equation}
R_{2,3,4}({\bf 5})=X_{1}X_{2}X_{3}+X_{2}X_{3}X_{4}+X_{3}X_{4}X_{1}+X_{4}X_{1}X_{2}\label{eq:5}
\end{equation}
  and
\begin{equation}
R_{2}(n)=X_{1}X_{2}+X_{2}X_{3}+\cdots+X_{n-1}X_{n}+X_{n}X_{1},\label{eq:6}
\end{equation}
where the indices are taken modulo n or are variant under circular
translation of indices (see formal definition chapter 2), and the
second function is the trapezoidal function. An example of a trapezoidal
function is given by
\[
\tau_{7,3}=X_{1}X_{2}X_{3}+X_{2}X_{3}X_{4}+X_{3}X_{4}X_{5}+X_{4}X_{5}X_{6}+X_{5}X_{6}X_{7}
\]
The name trapezoid comes from counting the number of times each variable
appears in the function $\tau_{n,k}$. For example, consider $\tau_{7,3}$.
Observe that $X_{1}$ appears 1 time in $\tau_{7,3}$, $X_{2}$ appears
2 times, $X_{3}$, $X_{4}$ and $X_{5}$ appears 3 times each, $X_{6}$
appears twice, and $X_{7}$ appears once. Plotting the these values
and connecting the dots produces the shape of an isosceles trapezoid.
Figure \ref{trap73} is a graphical representation of this. The Boolean
variable $X_{i}$ is represented by $i$ in the $x$-axis. The $y$-axis
corresponds to the number of times the variable appears in $\tau_{7,3}$.
\begin{figure}[h!]
\centering \includegraphics[width=3in]{\string"../../Subject on Thesis PHD/About on Thesis/On RotBF submmit Combinatorica - marzo 6 de 2017/Trapezoid73\string".eps}
\caption{Trapezoid associated to the Boolean function $\tau_{7,3}$}
\label{trap73} 
\end{figure}
. Both $\left\{ \tau_{n,3}\right\} _{n\geq3}$ and $\left\{ R_{2,3,4}\left(n\right)\right\} _{n\geq4}$
can be shown to satisfy a linear recurrence (see theorems \ref{trapezoidTHM}
and \ref{RotaBooleanGeneral}) using the following elementary method
, the $ON$ or $OFF$ method that we define below:
\begin{description}
\item [{Method~Boolean~case~($ON$~or~$OFF$)}] This technique consists
of the simple game of assigning the numeric value to the Boolean variable
$X_{i}$ a \textquotedbl{}$0$\textquotedbl{} or a \textquotedbl{}$1$\textquotedbl{},
that is, it is defined {\em $ON$: = turn on,} if the variable
$X_{i}=1$ and {\em $OFF$: = turn off, } if the variable $X_{i}=0$.
\end{description}
Using the assertion that the sequence $\left\{ \tau_{n,3}\right\} _{n\geq3}$
satisfy a linear recurrence with integer coefficients homogeneous
with $\tau_{n,3}$ defined over $\mathbb{F}_{3}$ and whose polynomial
associated with the homogeneous linear sequence of $\left\{ \tau_{n,3}\right\} _{n\geq3}$
(theorem $\left(\ref{thmtrapgen}\right)$) is given by 

\[
P_{3}\left(X\right)=X^{3}-3X-6,
\]
Now by the equation $\left(\ref{eq:2.3.4}\right)$ of the Chapter
$2$, the sequence$\left\{ \tau_{n,3}\right\} _{n\geq3}$ satisfies
the following linear recurrence
\[
\tau_{n,3}=3\tau_{n-1,3}+6\tau_{n-2,3}.
\]
Also by Theorem $\left(\ref{thmrot2}\right)$ the sequence $\left\{ R_{2}\left(n\right)\right\} _{n\geq2}$
also satisfies a homogeneous linear recurrence with integer coefficients
whose characteristic polynomial is
\[
P_{4}\left(X\right)=X^{4}-9=\left(x-3^{\frac{1}{2}}\right)\left(x+3^{\frac{1}{2}}\right)\left(x-3^{\frac{1}{2}}i\right)\left(x+3^{\frac{1}{2}}i\right),
\]
where $i=\sqrt{-1}$ . For properties of homogeneous linear recurrences,
there are constants $a_{1},a_{2},a_{3},a_{4}$ such that
\[
R_{2}\left(n\right)=a_{1}3^{\frac{n}{2}}+a_{2}\left(-3\right)^{\frac{n}{2}}+a_{3}\left(3^{\frac{1}{2}}i\right)^{n}+a_{4}\left(-3^{\frac{1}{2}}i\right)^{n}
\]
To conclude this section with this brief introduction a new concept
called the recursive generated set for $\left\{ s_{n}\right\} $ (see
definition $\left(\ref{RecurGenSequence}\right)$) which is based
the argument of the demonstrations that the sequence
\begin{equation}
\{S_{\mathbb{F}_{q}}\left(R_{2,3,\cdots,k}\left(n\right)\right)\}_{n\geq k}\,\mathnormal{\,\,\textnormal{and}\,\,}\,\{S_{\mathbb{F}_{q}}\left(e_{n,k}\right)\}_{n\geq k}
\end{equation}
satisfy homogeneous linear recurrences with integer coefficients.
Some well-known examples of exponential sums are special cases of
definition (\ref{expsumfq}).

\section{Closed formulas for some special exponential sum}

The closed formulas of exponential sums associated with the elementary
symmetric polynomial defined over field of galois are the central
axis in the research presented in chapter $3$. It is well known in
many areas of the mathematics that the tool of discrete fourier transform
is a mathematical object used for fast and efficient calculations.
In this thesis, chapter 3, presents how to obtain a characteristic
polynomial of a linear recurrence homogeneous of exponential sums
of symmetric polynomials defined over Galois field. In the same way
as shown, in the equation (\ref{eq:9}), Boolean case. For example,
the theorem $3.1$ in \cite{cm1} is a general Boolean version of
result of Cai, Green, and Thierauf. Similar to equation $\left(\ref{eq:9}\right)$,
let $1\leq k_{1}<\cdots<k_{s}$ be fixed integers and $r=\lfloor\log_{2}(k_{s})\rfloor+1$.
The value of the exponential sum 
\[
S(\boldsymbol{e}_{n,[k_{1},\cdots,k_{s}]})=S(\boldsymbol{e}_{n,k_{1}})+\cdots+S(\boldsymbol{e}_{n,k_{s}})
\]
 is given by
\begin{equation}
S(\boldsymbol{e}_{n,[k_{1},\cdots,k_{s}]})=c_{0}2^{n}+\sum_{j=1}^{2^{r}-1}c_{j}\lambda_{j}^{n},\label{eq:1.3.1}
\end{equation}
where $\lambda_{j}=1+\zeta_{j}$ , $\zeta_{j}=e^{\frac{\pi i\,j}{2^{r-1}}},\,\,i=\sqrt{-1}$
and $c_{j}$ are constant elements. The formula \ref{eq:1.3.1} is
a closed formula for exponential sum $S(\boldsymbol{e}_{n,[k_{1},\cdots,k_{s}]})$. 

The formula in equation (\ref{eq:1.3.1}) implicitly depends on the
discrete fourier transform to reach it (see \cite{cai,cm1}). The
implementation of these formulas in any mathematics program allows
the calculation of the value of the exponential sum $S(\boldsymbol{e}_{n,[k_{1},\cdots,k_{s}]})$
to be efficient in any computer. Once again, the discrete fourier
transform and mathematical object of circulat matrices are the basis
for obtaining the closed formulas of $S(\boldsymbol{e}_{n,k})$ for
any Gaois field (see chapter $3$). The closed formula of $S(\boldsymbol{e}_{n,k})$
allows us to show that the sequence $\left\{ S(\boldsymbol{e}_{n,k})\right\} $
satisfies homogeneous linear recurrences with integer coefficients
(see \cite{ccms}). Also in chapter $3$, the Theorem $\left(\ref{expsumformthm}\right)$
whose implementation in an old computer (whose features are not top
of the art) of the research, it took \textit{$Mathematica$} $0.008$seconds
to calculate 
\begin{equation}
S_{\mathbb{F}_{3}}(\boldsymbol{e}_{12,5})=346113+92664e^{\frac{2i\pi}{3}}+92664e^{-\frac{2i\pi}{3}}=253449.
\end{equation}
In comparison, it took 26.6 minutes when using the definition of the
exponential sum. The same implementation can be used to obtain values
of exponential sums for $n$ relatively big. For instance, it took
\textit{Mathematica} $1.28$ seconds to calculate 
\begin{equation}
\textsc{\ensuremath{S_{\mathbb{F}_{3}}}(\ensuremath{\boldsymbol{e}_{100,7}})=113935090835950800739864834563949291416514642941,}
\end{equation}
and $41.28$ seconds to calculate 
\begin{equation}
S_{\mathbb{F}_{4}}(\boldsymbol{e}_{50,5})=158735097466874432874732322816.
\end{equation}
It took about two minutes and a half to calculate $S_{\mathbb{F}_{3}}(\boldsymbol{e}_{500,11})$,
which is an integer with $239$ digits. For the Boolean case, the
study of the equation (\ref{eq:1.3.1}) leads us to the demonstration
of the cusick conjecture, Li and Stanica asymptotically (see \cite{cm1}).
It is clear that the computation of the values of $S_{\mathbb{F}_{q}}\left(F\right)$
is an exponentially hard problem, moreover, still the study of the
closed formula for $S_{\mathbb{F}_{q}}\left(\boldsymbol{e}_{n,k}\right)$
remains as an open research problem: the verification of the Cusick,
Li, Stanica conjecture asymptotically for fields Galois (see \cite{acgmr}).
On the other hand, the closed formula (see theorem \ref{expsumformthm})
shows a link between exponential sums of symmetric polynomials over
Galois fields and a problem for multinomial coefficients which is
similar to the problem of bisecting binomial coefficients. The elementary
symmetric Boolean polynomial $\boldsymbol{e}_{n,k}$ can be represented
as $\binom{i}{k}$ where $wt(X)=j$ is the Hamming weight of $X$,
then
\begin{equation}
S(\boldsymbol{e}_{n,k})=\sum_{j=0}^{n}(-1)^{\binom{j}{k}}\binom{n}{j}.
\end{equation}
This last equation is related to two Boolean mathematical problems
of interest in cryptography: the balanced of a Boolean function and
the problem of bisection of the binomial coefficients (see \cite{cusick2,cusick2.1,mitchell}).
A solution $(\delta_{0},\delta_{1},\cdots,\delta_{n})$ to the equation
\begin{equation}
\sum_{j=0}^{n}\delta_{j}\binom{n}{j}=0,\,\,\,\delta_{j}\in\{-1,1\},\label{bisec}
\end{equation}
is said to give a bisection of the binomial coefficients $\binom{n}{j}$,
$0\leq j\leq n.$ 

To finish this introduction the formula closed for $S_{\mathbb{F}_{q}}(\boldsymbol{e}_{n,k})$
is not only to computational improvement over the formal definition
of $S_{\mathbb{F}_{q}}(F)$, but also provide a link to a similar
problem to the problem of bisecting binomial coefficients (for multinomial
coefficients). For example. suppose that $\mathbb{F}_{3}=\mathbb{Z}_{3}$
is the Galois field of $3$ elements. Then a explicit formula for
$S_{\mathbb{F}_{3}}(\boldsymbol{e}_{4,2})$ is given by
\[
S_{\mathbb{F}_{3}}(\boldsymbol{e}_{4,2})=\sum_{m_{1}=0}^{4}\sum_{m_{2}=0}^{4-m_{1}}{4 \choose m_{0}^{*},m_{1},m_{2}}e^{\left(\frac{2\pi i}{3}\right)\left(\Lambda_{1,-1}(2,m_{1},m_{2})\right)}
\]
where $\Lambda_{1,-1}(2,m_{1},m_{2})=\sum_{j=0}^{m_{2}}{\text{\ensuremath{m_{2}}} \choose j}{\text{\ensuremath{m_{1}}} \choose 2-j}\left(-1\right)^{j}$
and $m_{0}^{*}=4-(m_{1}+m_{2})$. Moreover, the fact that exponential
sums of symmetric polynomials over finite fields can be expressed
as multinomial sums is later used in the proof of closed formulas
for them. The proof of the closed formulas also depends on a classical
result in number theory known as Lucas' Theorem. 

\chapter{Recursion}

A Boolean function is a function from the vector space $\mathbb{F}_{2}^{n}$
to $\mathbb{F}_{2}$. Boolean functions are part of a beautiful branch
of combinatorics with applications to many scientific areas. Some
particular examples are the areas of theory of error-correcting codes
and cryptography. Efficient cryptographic implementations of Boolean
functions with many variables is a challenging problem due to memory
restrictions of current technology. Because of this, symmetric Boolean
functions are good candidates for efficient implementations. However,
symmetry is too special a property and may imply that these implementations
are vulnerable to attacks.

%%%%%%%%%%%%%%%%%%%%%%%%%%%%%%%%%%%%%%%%%%%%%%%%%%%%%%%%%%%%%%%%%%%%%%%%%%%%%%%%%%%%%%%%%%

\section{Preliminaries}

As mentioned in the introduction, Pieprzyk and Qu (\cite{piequ})
introduced rotation symmetric Boolean functions. A \textit{rotation
symmetric Boolean function} in $n$ variables is a function which
is invariant under the action of the cyclic group $C_{n}$ on the
set $\mathbb{F}_{2}^{n}$. Let us explain this definition in a more
concrete way. Our explanation is similar to the one presented in \cite{stanicamaitra}.

Let $1<j_{1}<\cdots<j_{s}$ be integers. A rotation symmetric Boolean
function of the form 
\begin{equation}
R_{j_{1},\cdots,j_{s}}(n)=X_{1}X_{j_{1}}\cdots X_{j_{s}}+X_{2}X_{j_{1}+1}\cdots X_{j_{s}+1}+\cdots+X_{n}X_{j_{1}-1}\cdots X_{j_{s}-1},\label{monrot}
\end{equation}
where the indices are taken modulo $n$ and the complete system of
residues is $\{1,2,\cdots,n\}$, is called a \textit{(long cycle)
monomial rotation symmetric} Boolean function. For example, the rotation
symmetric Boolean function (\ref{rotex}) is given by 
\begin{equation}
R({\bf X})=R_{2,3}(5)+R_{3}(5).
\end{equation}
Sometimes the notation $(1,j_{1},\cdots,j_{s})_{n}$ is used to represent
the monomial rotation Boolean function (\ref{monrot}), see \cite{cusickArXiv}.

As mentioned in the introduction, in this work we present a method
that could be used to generalize Cusick's result over any Galois field.
In particular, we show that exponential sums over finite fields of
some rotation symmetric polynomials are linear recurrent with integer
coefficients. The \textit{exponential sum} of a function $F:\mathbb{F}_{q}^{n}\to\mathbb{F}_{q}$
is given by 
\begin{equation}
S_{\mathbb{F}_{q}}(F)=\sum_{{\bf x}\in\mathbb{F}_{q}^{n}}e^{\frac{2\pi i}{p}\text{Tr}_{\mathbb{F}_{q}/\mathbb{F}_{p}}(F({\bf x}))}.\label{expsumfq}
\end{equation}
Here, $\text{Tr}_{\mathbb{F}_{q}/\mathbb{F}_{p}}$ represents the
\textit{field trace function} from $\mathbb{F}_{q}$ to $\mathbb{F}_{p}$.

Exponential sums are very rich objects in the area of analytic number
theory. 

In the next section we provide an introduction to the elementary method
used in this article to obtain linear recurrences for this type of
exponential sums. As mentioned before, this introduction is done over
$\mathbb{F}_{2}$. The reader interested in the generalization is
invited to skip this section and go directly to section \ref{anyGalois}.

%%%%%%%%%%%%%%%%%%%%%%%%%%%%%%%%%%%%%%%%%%%%%%%%%%%%%%%%%%%%%%%%%%%
% Linear recurrences over F2
%%%%%%%%%%%%%%%%%%%%%%%%%%%%%%%%%%%%%%%%%%%%%%%%%%%%%%%%%%%%%%%%%%%

\section{Linear recurrences over $\mathbb{F}_{2}$}

We start the discussion with the recurrence for exponential sums of
trapezoid Boolean functions.
\begin{thm}
\label{trapezoidTHM} The sequence $\{S(\tau_{n,k})\}_{n=k}^{\infty}$
satisfies a homogeneous linear recurrence with integer coefficients
whose characteristic polynomial is given by 
\begin{equation}
p_{k}(X)=X^{k}-2(X^{k-2}+X^{k-3}+\cdots+X+1).\label{charpoly}
\end{equation}
\end{thm}

\begin{rem}
We point out that a more general version of Theorem \ref{trapezoidTHM}
for cubic functions appears in \cite{browncusick}. 
\end{rem}

the same arguments as in the proof of Theorem \ref{trapezoidTHM}.
\begin{lem}
\label{trapezoidplusLemma} Let $\tau_{n,k}$ be the trapezoid Boolean
function of degree $k$ in $n$ variables. Suppose that $F({\bf X})$
is a Boolean polynomial in the first $j$ variables with $j<k$. Then,
the sequences 
\[
\{S(\tau_{n,k}+F({\bf X}))\}
\]
and 
\[
\{S(\tau_{n,k}+F({\bf X})+X_{n}+X_{n}X_{n-1}+X_{n}X_{n-1}X_{n-2}+\cdots+X_{n}X_{n-1}\cdots X_{n-k+2})\}
\]
satisfies the linear recurrence whose characteristic polynomial is
given by $p_{k}(X)$. 
\end{lem}

Theorem \ref{trapezoidTHM} and Lemma \ref{trapezoidplusLemma} are
all that is needed to show that the sequence of exponential sums of
$R_{2,3,\cdots,k}(n)$ satisfies the linear recurrence with characteristic
polynomial $p_{k}(X)$.
\begin{thm}
\label{RotaBooleanGeneral}The sequence $\{S(R_{2,3,\cdots,k}(n))\}$
satisfies the homogeneous linear recurrence whose characteristic polynomial
is given by $p_{k}(X)$. 
\end{thm}

%%%%%%%%%%%%%%%%%%%%%%%%%%%%%%%%%%%%%%%%%%%%%%%%%%%%%%%%%%%%%%%%%%%
% Linear recurrences over Fq
%%%%%%%%%%%%%%%%%%%%%%%%%%%%%%%%%%%%%%%%%%%%%%%%%%%%%%%%%%%%%%%%%%%

\section{Linear recurrences over $\mathbb{F}_{q}$}

In this section we show that Cuscik's result is not unique to the
Boolean case. In fact, exponential sums over finite fields of rotation
polynomials $R_{j_{1},\cdots,j_{s}}(n)$ (and linear combination of
them) satisfy linear recurrences with constant coefficients. This
is a generalization of Cusick's result.

Consider the Galois field $\mathbb{F}_{q}=\{0,\alpha_{1},\cdots,\alpha_{q-1}\}$
where $q=p^{r}$ with $p$ prime and $r\geq1$. The recall that the
exponential sum of a function $F:\mathbb{F}_{q}^{n}\to\mathbb{F}_{q}$
is given by 
\begin{equation}
S_{\mathbb{F}_{q}}(F)=\sum_{{\bf x}\in\mathbb{F}_{q}^{n}}e^{\frac{2\pi i}{p}\text{Tr}_{\mathbb{F}_{q}/\mathbb{F}_{p}}(F({\bf x}))},
\end{equation}
where $\text{Tr}_{\mathbb{F}_{q}/\mathbb{F}_{p}}$ represents the
field trace function from $\mathbb{F}_{q}$ to $\mathbb{F}_{p}$.
The same technique used for exponential sums of Boolean functions
can be used in general. However, instead of having two options for
the ``switch\textquotedbl{}, we now have $q$ of them. Let $X$ be
a variable which takes values on $\mathbb{F}_{q}$. As before, we
say that the variable $X$ can be turned \textit{OFF} or \textit{ON},
however, this time the term ``turn \textit{OFF}\textquotedbl{} means
that $X$ assumes the value 0, while the term ``turn \textit{ON}\textquotedbl{}
means that $X$ assumes all values in $\mathbb{F}_{q}$ that are different
from zero. Think of this situation as a light switch on which you
have the option to turn \textit{OFF} the light and the option to turn
it \textit{ON} to one of $q-1$ different colors.

We consider first sequences of exponential sums of trapezoid functions.
As in the Boolean case, they satisfy linear recurrences with integer
coefficients over any Galois field $\mathbb{F}_{q}$. We start with
the following lemma, which is interesting in its own right.
\begin{lem}
\label{generallemma} Let $k,n$ and $j$ be integers with $k>2$,
$1\leq j<k$ and $n\geq k$. Then, 
\begin{equation}
S_{\mathbb{F}_{q}}\left(T_{2,3,\cdots,k}(n)+\sum_{s=1}^{j}\beta_{s}\prod_{l=0}^{k-s-1}X_{n-l}\right)=S_{\mathbb{F}_{q}}\left(T_{2,3,\cdots,k}(n)+\sum_{s=1}^{j}\prod_{l=0}^{k-s-1}X_{n-l}\right)
\end{equation}
for any choice of $\beta_{s}\in\mathbb{F}_{q}^{\times}$. 
\end{lem}

Next is the linear recurrence for exponential sums of trapezoid functions
over any Galois field.
\begin{thm}
\label{thmtrapgen} Let $k\geq2$ be an integer and $q=p^{r}$ with
$p$ prime. The sequence $\{S_{\mathbb{F}_{q}}(T_{2,3,\cdots,k}(n))\}_{n=k}^{\infty}$
satisfies a homogeneous linear recurrence with integer coefficients
whose characteristic polynomial is given by 
\begin{equation}
Q_{T,k,\mathbb{F}_{q}}(X)=X^{k}-q\sum_{l=0}^{k-2}(q-1)^{l}X^{k-2-l}.\label{charpolygeneral}
\end{equation}
In particular, when $q=2$ we recover Theorem \ref{trapezoidTHM}. 
\end{thm}

\begin{proof}
M

\begin{equation}
a_{n+1}=\sum_{l=0}^{k-2}q(q-1)^{l}a_{n-1-l}\label{eq:2.3.4}
\end{equation}
\end{proof}
m
\begin{thm}
Let $f(x)=a_{n}x^{n}+a_{n-1}x^{n-1}+\cdots+a_{1}x+a_{0}\in\mathbb{Z}[x]$
be a polynomial. Let $p$ be a prime. Denote the $p$-adic valuation
of an integer $m$ by $\nu_{p}(m)$ (with $\nu_{p}(0)=+\infty$).
Suppose that 
\end{thm}

\begin{enumerate}
\item $\nu_{p}(a_{n})=0$, 
\item $\nu_{p}(a_{n-i})/i>\nu_{p}(a_{0})/n$ for $1\leq i\leq n-1$, and 
\item $\gcd(\nu_{p}(a_{0}),n)=1$. 
\end{enumerate}
Then, $f(x)$ is irreducible over $\mathbb{Q}$. 
\begin{prop}
Let $q=p^{r}$ with $p$ prime. Suppose that $\gcd(k,r)=1$. Then,
the polynomial 
\begin{equation}
Q_{T,k,\mathbb{F}_{q}}(X)=X^{k}-q\sum_{l=0}^{k-2}(q-1)^{l}X^{k-2-l}
\end{equation}
is irreducible over $\mathbb{Q}$. 
\end{prop}

\begin{thm}
\label{thmrot2} Suppose that $p>2$ is prime. Then, $\{S_{\mathbb{F}_{p}}(R_{2}(n)\}$
satisfy the homogeneous linear recurrence with characteristic polynomial
\begin{equation}
Q_{R,2,\mathbb{F}_{p}}(X)=X^{4}-p^{2}.
\end{equation}
\end{thm}

M
\begin{thm}
\label{generalTHM} Let $k\geq2$ be an integer and $q=p^{r}$ with
$p$ prime and $r\geq1$. The sequence $\{S_{\mathbb{F}_{q}}(R_{2,3,\cdots,k}(n))\}_{n\geq k}$
satisfies a linear recurrence with integer coefficients. 
\end{thm}

\begin{defn}
\label{RecurGenSequence}Let $\{b(n)\}$ be a sequence on an integral
domain $D$. A set of sequences 
\[
\{\{a_{1}(n)\},\{a_{2}(n)\},\cdots,\{a_{s}(n)\}\},
\]
where $s$ is some natural number, is called a {\em recursive generating
set for} $\{b(n)\}$ if 
\end{defn}

\begin{enumerate}
\item there is an integer $l$ such that for every $n$, $b(n)$ can be
written as a linear combination of the form 
\[
b(n)=\sum_{j=1}^{s}c_{j}\cdot a_{j}(n-l),
\]
where $c_{j}$'s are constants that belong to $D$, and 
\item for each $1\leq j_{0}\leq s$ and every $n$, $a_{j_{0}}(n)$ can
be written as a linear combination of the form 
\[
a_{j_{0}}(n)=\sum_{j=1}^{s}d_{j}\cdot a_{j}(n-1),
\]
where $d_{j}$'s are also constants that belong to $D$. 
\end{enumerate}
The sequences $\{a_{j}(n)\}$'s are called \textit{recursive generating
sequences for} $\{b(n)\}$. 
\begin{rem}
It is a well-known result in the theory of recursive sequences that
a sequence that has a recursive generating set satisfies a linear
recurrence with constant coefficients. In fact, this technique has
been used in Theorems \ref{thmrot2} and \ref{generalTHM}. 
\end{rem}

\begin{thm}
Let $p$ be prime and $q=p^{r}$. Consider the function 
\[
F_{n}=\sum_{t=1}^{N}\beta_{t}R_{j_{t,1},\cdots,j_{t,s_{t}}}(n),
\]
where $\beta_{t}\in\mathbb{F}_{q}$ and $1<j_{t,1}<\cdots<j_{t,s_{t}}$
are integers. The sequence $\{S_{\mathbb{F}_{q}}(F_{n})\}$ satisfies
a linear recurrence with integer coefficients. 
\end{thm}

M
\begin{thm}
\label{MoregeneralTHMSymmetric} Let $k\geq2$ be an integer and $q=p^{r}$
with $p$ prime and $r\geq1$. The sequence 
\begin{equation}
\left\{ S_{\mathbb{F}_{q}}\left(\sum_{j=0}^{k-1}\beta_{j}\sigma_{n,k-j}\right)\right\} 
\end{equation}
satisfies a linear recurrence with constant coefficients, regardless
of the choice of the $\beta_{j}$'s. 
\end{thm}

M
\begin{thm}
\label{eigenthm} Let $C(p)$ be the set of eigenvalues of $M(p)$.
Let $\zeta_{p}=e^{2\pi i/p}$. Then, $\lambda\in C(p)$ if and only
if In particular, $|C(p)|=(p+1)/2$. 
\end{thm}

%%%%%%%%%%%%%%%%%%%%%%%%%%%%%%%%%%%%%%%%%%%%%%%%%%%%%%%%%%%%%%%%%%%%%
% Concluding remarks
%%%%%%%%%%%%%%%%%%%%%%%%%%%%%%%%%%%%%%%%%%%%%%%%%%%%%%%%%%%%%%%%%%%%%

\section{Concluding remarks}

This article is divided as follows. The next section contains some
preliminaries. In Section \ref{multinomialsums} we provide multinomial
sum expressions for exponential sums of symmetric polynomials over
Galois fields. We also include some representations that depend on
integer partitions. These multinomial sums representations are a computational
improvement over the formal definition of exponential sums. Moreover,
as just mentioned, they provide a connection to a problem similar
to the problem of bisecting binomial coefficients. Section \ref{closedformulas}
is the core and final section of this article. It is also the section
where the main results are presented. In particular, we find closed
formulas for some multinomial sums. This, together with multinomial
sum representations for our exponential sums, allow us to prove closed
formulas for exponential sums of linear combinations of elementary
symmetric polynomials over finite fields. We also provide explicit
linear recurrences for such exponential sums, showing that the recursive
nature of these sequences is not special to the binary case. Moreover,
every multi-variable function over a finite field extension of $\mathbb{F}_{2}$
can be identified with a Boolean function. Thus, these results also
provide new families of Boolean functions that might be useful for
efficient implementations.

%%%%%%%%%%%%%%%%%%%%%%%%%%%%%%%%%%%%%%%%%%%%%%%%%%%%
% Preliminaries
%%%%%%%%%%%%%%%%%%%%%%%%%%%%%%%%%%%%%%%%%%%%%%%%%%%%

\chapter{Closed formulas for exponential sums of symmetric polynomial}

\section{Preliminaries}

It is a well-established result in the theory of Boolean functions
that any symmetric Boolean function can be identified with a linear
combination of elementary symmetric Boolean polynomials. To be more
precise, let $\boldsymbol{e}_{n,k}$ be the elementary symmetric polynomial
in $n$ variables of degree $k$. For example, 
\[
\boldsymbol{e}_{4,3}=X_{1}X_{2}X_{3}\oplus X_{1}X_{4}X_{3}\oplus X_{2}X_{4}X_{3}\oplus X_{1}X_{2}X_{4},
\]
where $\oplus$ represents addition modulo 2. Every symmetric Boolean
function $F({\bf X})$ can be identified with an expression of the
form where $0\leq k_{1}<k_{2}<\cdots<k_{s}$ are integers. For the
sake of simplicity, the notation $\boldsymbol{e}_{n,[k_{1},\ldots,k_{s}]}$
is used to denote (\ref{genboolsym}). For example, 
\begin{thm}
\label{invariant} Let $1\leq k_{1}<\cdots<k_{s}$ be fixed integers
and $r=\lfloor\log_{2}(k_{s})\rfloor+1$. The value of the exponential
sum $S(\boldsymbol{e}_{n,[k_{1},\cdots,k_{s}]})$ is given by 
\begin{eqnarray*}
S(\boldsymbol{e}_{n,[k_{1},\cdots,k_{s}]}) & = & c_{0}(k_{1},\cdots,k_{s})2^{n}+\sum_{j=1}^{2^{r}-1}c_{j}(k_{1},\cdots,k_{s})(1+\zeta_{j})^{n},
\end{eqnarray*}
where $\zeta_{j}=e^{\frac{\pi i\,j}{2^{r-1}}},i=\sqrt{-1}$ and 
\begin{equation}
c_{j}(k_{1},\cdots,k_{s})=\frac{1}{2^{r}}\sum_{t=0}^{2^{r}-1}(-1)^{\binom{t}{k_{1}}+\cdots+\binom{t}{k_{s}}}\zeta_{j}^{-t}.\label{coeffs}
\end{equation}
\end{thm}

Theorem \ref{invariant} and a closed formula for $c_{0}(k)$ (proved
in \cite{cm1}) were used by Castro and Medina \cite{cm1} to prove
asymptotically a conjecture of Cusick, Li and St$\check{\mbox{a}}$nic$\check{\mbox{a}}$
about the balancedness of elementary symmetric polynomials \cite{cusick2}.
An adaptation of Theorem \ref{invariant} to perturbations of symmetric
Boolean functions (see \cite{cm2}) was recently used in \cite{cgm2}
to prove a generalized conjecture of Canteaut and Videau \cite{canteaut}
about the existence of balanced perturbations when the number of variables
grows. The original conjecture, which was stated for symmetric Boolean
functions, said that only trivially balanced functions exists when
the number of variables grows. The original conjecture was proved
by Guo, Gao and Zhao \cite{ggz}. The same behavior holds true for
perturbations of symmetric Boolean functions.

One of the goals of this article is to generalize Theorem \ref{invariant}
to the general setting of Galois fields. If $F:\mathbb{F}_{q}^{n}\to\mathbb{F}_{q}$,
then its \textit{exponential sum over $\mathbb{F}_{q}$} is given
by 
\begin{equation}
S_{\mathbb{F}_{q}}(F)=\sum_{{\bf x}\in\mathbb{F}_{q}^{n}}e^{\frac{2\pi i}{p}\text{Tr}_{\mathbb{F}_{q}/\mathbb{F}_{p}}(F({\bf x}))},\label{expSq}
\end{equation}
trace function from $\mathbb{F}_{q}$ to $\mathbb{F}_{p}$. The \textit{field
trace function} can be explicitly defined as 
\begin{equation}
\text{Tr}_{\mathbb{F}_{p^{l}}/\mathbb{F}_{p}}(\alpha)=\sum_{j=0}^{l-1}\alpha^{p^{j}},
\end{equation}
with arithmetic done in $\mathbb{F}_{p^{l}}$. Recently in \cite{ccms},
it was proved that exponential sums over $\mathbb{F}_{q}$ of linear
combinations of elementary symmetric polynomials are linear recurrent
with integer coefficients. Thus, the recursive nature of these sequences
is not restricted to $\mathbb{F}_{2}$. The approach presented in
\cite{ccms}, however, 
\begin{thm}
Suppose that $n$ and $k$ are non-negative integers and let $p$
be a prime. Suppose that 
\begin{eqnarray*}
n & = & n_{0}+n_{1}p+\cdots+n_{l}p^{l}\\
k & = & k_{0}+k_{1}p+\cdots+k_{l}p^{l},
\end{eqnarray*}
with $0\leq n_{j},k_{j}<p$ for $j=1,\cdots,l$. Then, 
\[
\binom{n}{k}\equiv\prod_{j=0}^{l}\binom{n_{j}}{k_{j}}\mod p.
\]
\end{thm}

Let $D=p^{\lfloor\log_{p}(k)\rfloor+1}$. Observe that one consequence
of Lucas' Theorem is 
\begin{equation}
\binom{n+D}{k}\equiv\binom{n}{k}\mod p.
\end{equation}
This will be used throughout the rest of the paper.

%%%%%%%%%%%%%%%%%%%%%%%%%%%%%%%%%%%%%%%%%%%%%%%%%%%%%%%%%%%%%%%%%%%%%%%%%%%%%%
% A formula for exponential sums in terms of multinomial coefficients
%%%%%%%%%%%%%%%%%%%%%%%%%%%%%%%%%%%%%%%%%%%%%%%%%%%%%%%%%%%%%%%%%%%%%%%%%%%%%%

\section{A formula for exponential sums in terms of multinomial sums}

\label{multinomialsums}

In this section we prove a formula for $S_{\mathbb{F}_{q}}(\boldsymbol{e}_{n,k})$
in terms of multinomial coefficients. This formula is a computational
improvement over (\ref{expSq}). We start by finding a formula, in
this case, a recursive one, for the value of $\boldsymbol{e}_{n,k}$
at a vector ${\bf x}$.

Let $n,k$ and $m$ be positive integers and $a_{s}$ be a parameter
($s$ a positive integer). Let 
\begin{equation}
\Lambda_{a_{1}}(k,m)=a_{1}^{k}\binom{m}{k}
\end{equation}
and define $\Lambda_{a_{1},\cdots,a_{l}}$ recursively by 
\begin{equation}
\Lambda_{a_{1},a_{2},\cdots,a_{l+1}}(k,m_{1},m_{2},\cdots,m_{l+1})=\sum_{j=0}^{m_{l+1}}\binom{m_{l+1}}{j}a_{l+1}^{j}\Lambda_{a_{1},\cdots,a_{l}}(k-j,m_{1},m_{2},\cdots,m_{l}),
\end{equation}
The value of $\boldsymbol{e}_{n,k}$ is linked to $\Lambda_{a_{1},\cdots,a_{l}}$.
\begin{lem}
\label{lemmaSymm} Let $n$ and $k$ be positive integers. Let $A_{l}=\{0,a_{1},\cdots,a_{l}\}$
and ${\bf x}\in A_{l}^{n}$. Suppose that $a_{j}$ appears $m_{j}$
times in ${\bf x}$. Then, 
\begin{equation}
\boldsymbol{e}_{n,k}({\bf x})=\Lambda_{a_{1},\cdots,a_{l}}(k,m_{1},\cdots,m_{l}).
\end{equation}
\end{lem}

The above lemma can be used to express exponential sums of symmetric
polynomials as a multi-sum of products of multinomial coefficients.
\begin{thm}
\label{expsumformthm} Let $n,k$ be natural numbers such that $k\leq n$,
$p$ a prime and $q=p^{r}$ for some positive integer $r$. Suppose
that $\mathbb{F}_{q}=\{0,\alpha_{1},\cdots,\alpha_{q-1}\}$ is the
Galois field of $q$ elements. Then, where $m_{0}^{*}=n-(m_{1}+\cdots+m_{q-1})$. 
\end{thm}

An easy adjustment to the proof of Theorem \ref{expsumformthm} leads
the following corollary.
\begin{cor}
\label{expsumformcoro} Let $1\leq k_{1}<k_{2}<\cdots<k_{s}$ and
$n$ be positive integers, $p$ a prime and $q=p^{r}$ for some positive
integer $r$. Suppose that $\mathbb{F}_{q}=\{0,\alpha_{1},\cdots,\alpha_{q-1}\}$
is the Galois field of $q$ elements. Consider the symmetric function
Then, 
\end{cor}

Theorem \ref{expsumformthm} and its corollary can be written in terms
of partitions of $n$. We say that $\boldsymbol{\lambda}=(\lambda_{1},\cdots,\lambda_{r})$
is a {\em partition} of $n$, and write $\boldsymbol{\lambda}\dashv n$,
if the $\lambda_{j}$ are integers and 
\[
\lambda_{1}\geq\cdots\geq\lambda_{r}\geq1\,\,\text{ and }\,\,n=\lambda_{1}+\cdots+\lambda_{r}.
\]
We us to denote the set of all rearrangements of $\boldsymbol{\lambda}$.
Finally, if $\boldsymbol{\gamma}$ is a non-empty list, then $\boldsymbol{\gamma}^{*}$
is the list obtained from $\boldsymbol{\gamma}$ by removing the first
element. For example, if $\boldsymbol{\gamma}=(2,2,1,1)$, then $\boldsymbol{\gamma}^{*}=(2,1,1)$.
Theorem \ref{expsumformthm} and Corollary \ref{expsumformcoro} can
be re-stated as follows. 
\begin{thm}
\label{thmpart} Let $n,k$ be natural numbers such that $k\leq n$,
$p$ a prime and $q=p^{r}$ for some positive integer $r$. Suppose
that $\mathbb{F}_{q}=\{0,\alpha_{1},\cdots,\alpha_{q-1}\}$ is the
Galois field of $q$ elements. Then, 
\end{thm}

\begin{cor}
\label{coropart} Let $1\leq k_{1}<k_{2}<\cdots<k_{s}$ and $n$ be
positive integers, $p$ a prime and $q=p^{r}$ for some positive integer
$r$. Suppose that $\mathbb{F}_{q}=\{0,\alpha_{1},\cdots,\alpha_{q-1}\}$
is the Galois field of $q$ elements. Consider the symmetric function
\[
\sum_{j=1}^{s}\beta_{j}\boldsymbol{e}_{n,k_{j}}\,\,\,\text{ where }\beta_{j}\in\mathbb{F}_{q}^{\times}.
\]
Then, 
\end{cor}

For small $q$, Theorem \ref{expsumformthm} and the recursive nature
of $\Lambda_{a_{1},\cdots,a_{l}}$ can be used to speed up the computation
of $S_{\mathbb{F}_{q}}(\boldsymbol{e}_{n,k})$. For example, using
an implementation of Theorem \ref{expsumformthm} and an old computer
(whose features are not top of the art) from one of the authors, it
took \textit{Mathematica} 0.008 seconds to calculate where the indices
run 
\[
0\leq m_{0}\leq n,0\leq m_{1}\leq n-m_{0},\cdots,0\leq m_{q-2}\leq n-m_{0}-m_{1}-\cdots-m_{q-3},
\]
into $p$ sublists, $l_{j}(n;q),1\leq j\leq p$, such that the sum
on each sublist is the same. This common sum must be $q^{n-1}$. Observe
that every time $S_{\mathbb{F}_{q}}(\beta_{1}\boldsymbol{e}_{n,k_{1}}+\cdots+\beta_{s}\boldsymbol{e}_{n,k_{s}})=0$
we obtain a $(p,q)$-section to of multinomial coefficients. This
connection generalizes the one that exists between bisections of binomial
coefficients and symmetric Boolean functions.
\begin{example}
The elementary symmetric polynomial $\boldsymbol{e}_{5,3}$ is such
that $S_{\mathbb{F}_{3}}(\boldsymbol{e}_{5,3})=0$. Observe that 
\end{example}

M
\begin{example}
The symmetric polynomial $\boldsymbol{e}_{6,5}+\boldsymbol{e}_{6,3}$
also satisfies $S_{\mathbb{F}_{3}}(\boldsymbol{e}_{6,5}+\boldsymbol{e}_{6,3})=0$.
In this case, 
\begin{equation}
\mathcal{L}(6;3)=\{1,6,15,20,15,6,1,6,30,60,60,30,6,15,60,90,60,15,20,60,60,20,15,30,15,6,6,1\}.
\end{equation}
The 3-section that corresponds to $\boldsymbol{e}_{6,5}+\boldsymbol{e}_{6,3}$
over $\mathbb{F}_{3}$ is 
\begin{eqnarray}
l_{1}(6;3) & = & \{1,6,6,15,15,20,30,30,30,90\}\label{3sec}\\
l_{2}(6;3) & = & \{1,6,6,15,15,20,60,60,60\}\nonumber \\
l_{3}(6;3) & = & \{1,6,6,15,15,20,60,60,60\}.\nonumber 
\end{eqnarray}
\end{example}

As in the Boolean case, we may try to define trivial $(p,q)$-sections.
A possible way to do this is to say that a $(p,q)$-section is trivial
if $l_{1}(n;k)=l_{2}(n;k)=\cdots=l_{p}(n;k)$. Again, following the
binary case, we say that a symmetric polynomial $\beta_{1}\boldsymbol{e}_{n,k_{1}}+\cdots+\beta_{s}\boldsymbol{e}_{n,k_{s}}$
is trivially balanced over $\mathbb{F}_{q}$ if its related $(p,q)$-section
is trivial. For example, $\boldsymbol{e}_{5,3}$ is trivially balanced,
while $\boldsymbol{e}_{6,5}+\boldsymbol{e}_{6,3}$ is not. It would
be interesting to know if some results known for the binary case also
apply to this problem.

Exponential sums of linear combinations of elementary symmetric polynomials
are also linked, via Theorem we find a solution to (\ref{diopheq}).
\begin{example}
Consider $\mathbb{F}_{4}=\{0,1,\alpha,\alpha+1\}$ where $\alpha^{2}=\alpha+1$.
The symmetric polynomial 
\[
(1+\alpha)\boldsymbol{e}_{n,3}+(1+\alpha)\boldsymbol{e}_{n,2}+\alpha\boldsymbol{e}_{n,1}
\]
is such that 

A natural problem to explore is to see how solutions to (\ref{diopheq})
given by exponential sums of linear combinations of elementary symmetric
polynomials look like as $n$ grows. Perhaps something similar to
the study presented in \cite{cgm2} holds true in this case. This
is part of future research.
\end{example}

In the next section, we prove closed formulas for exponential sums
of symmetric polynomials over Galois fields. Moreover, we provide
explicit linear recurrences with integer coefficients for these exponential
sums.

%%%%%%%%%%%%%%%%%%%%%%%%%%%%%%%%%%%%%%%%%%%%%%%%%%%%%%%%%%
% Closed formulas
%%%%%%%%%%%%%%%%%%%%%%%%%%%%%%%%%%%%%%%%%%%%%%%%%%%%%%%%%%

\section{Closed formulas for exponential sums of symmetric polynomials}

\label{closedformulas} In this section we generalize Theorem \ref{invariant},
that is, we provide closed formulas for the exponential sums considered
in this article. These formulas, in turn, allow us to find explicit
recursions for these sequences. Our formulas depend on circulant matrices
and on periodicity. Thus, we start with a short background on these
topics.

Let $D$ be a positive integer and $\alpha=\left(c_{0},c_{1},\ldots,c_{D-1}\right)\in\mathbb{C}^{D}$.
The $D$-{\em circulant matrix} associated to $\alpha$, denoted
by $\text{circ}(\alpha)$, is defined by

\begin{equation}
\text{circ}(\alpha):=\left(\begin{array}{ccccc}
c_{0} & c_{1} & \cdots & c_{D-2} & c_{D-1}\\
c_{D-1} & c_{0} & \cdots & c_{D-1} & c_{D-2}\\
\vdots & \vdots & \ddots & \vdots & \vdots\\
c_{2} & c_{3} & \cdots & c_{0} & c_{1}\\
c_{1} & c_{2} & \cdots & c_{D-1} & c_{0}
\end{array}\right).
\end{equation}
The polynomial $p_{\alpha}(X)=c_{0}+c_{1}X+\cdots+c_{D-1}X^{D-1}$
is called the {\em associated polynomial} of the circulant matrix.
In the literature, this polynomial is also called {\em representer
polynomial}. Observe that if 
\begin{prop}
\label{closedformprop} Let $n\in\mathbb{N}$ and $0\leq t\leq D-1$.
Then, 
\begin{equation}
r_{t}(n;a)=\frac{1}{D}\sum_{m=0}^{D-1}\xi_{D}^{tm}\lambda_{m}^{n},
\end{equation}
where $\xi_{D}=\exp(2\pi i/D)$ and $\lambda_{m}=1+a\xi_{D}^{-m}$
are the eigenvalues of 
\end{prop}

The following results are easy consequences of the above proposition.
\begin{cor}
\label{corocloseda} Let $F$ be a periodic function with period $D$.
Suppose that $\xi^{D}=1$ (not necessarily primitive). Then, 
\begin{equation}
\sum_{l=0}^{n}\binom{n}{l}a^{l}\xi^{F(l)}=\frac{1}{D}\sum_{t=0}^{D-1}\xi^{F(t)}\sum_{j=0}^{D-1}\xi_{D}^{tj}\lambda_{j}^{n},
\end{equation}
where $\xi_{D}=\exp(2\pi i/D)$ and $\lambda_{j}=1+a\xi_{D}^{-j}$,
for $0\leq j\leq D-1$, are the eigenvalues of 
\end{cor}

M
\begin{cor}
\label{coroclosed} Let $F$ be a periodic function with period $D$.
Suppose that $\xi^{D}=1$ (not necessarily primitive). Then, 
\begin{equation}
\sum_{l=0}^{n}\binom{n}{l}\xi^{F(l)}=\frac{1}{D}\sum_{t=0}^{D-1}\xi^{F(t)}\sum_{j=0}^{D-1}\xi_{D}^{tj}\left(1+\xi_{D}^{-j}\right)^{n},
\end{equation}
where $\xi_{D}=\exp(2\pi i/D)$. 
\end{cor}

These results can be extended further to obtain closed formulas for
multinomial sums.
\begin{thm}
\label{generalclosed} Let $F(q_{1},\cdots,q_{r})$ be a periodic
function in each component. Moreover, suppose that $D$ is a period
for $F$ in each component and that $\xi^{D}=1$ (not necessarily
primitive). Define, where $\xi_{D}=\exp(2\pi i/D)$ and $\lambda_{j_{1},\cdots,j_{r}}=1+\xi_{D}^{-j_{1}}+\xi_{D}^{-j_{2}}+\cdots+\xi_{D}^{-j_{r}}$.
\end{thm}

Observe that equation (\ref{nicerep}) can be written as where However,
note that $\lambda_{t_{1},\cdots,t_{r}}=\lambda_{t'_{1},\cdots,t'_{r}}$
where $(t'_{1},\cdots,t'_{r})$ is any rearrangement of $(t_{1},\cdots,t_{r})$.
This means that the coefficient of $\lambda_{t_{1},\cdots,t_{r}}^{n}$
in (\ref{nicerep}) is the sum of all $d_{t_{1}',\cdots,t_{r}'}(D)$
where $(t_{1}',\cdots,t_{r}')$ is a rearrangement of $(t_{1},\cdots,t_{r})$.
rearrangements of $(t_{1},\cdots,t_{r})$. Theorem \ref{generalclosed}
now can be re-stated as follows.
\begin{thm}
\label{generalclosedA} Let $F(q_{1},\cdots,q_{r})$ be a periodic
function in each component. Moreover, suppose that $D$ is a period
for $F$ in each component and that $\xi^{D}=1$ (not necessarily
primitive). Define, and $\xi_{D}=\exp(2\pi i/D)$.
\end{thm}

A nice consequence of this result is that sequences of the form $\{S(n)\}$,
with $S(n)$ defined as in (\ref{gensumA}), satisfy linear recurrences
with integer coefficients. Moreover, we can provide explicit characteristic
polynomials for such recurrences.
\begin{cor}
\label{gencoro} 
\end{cor}

The linear recurrence given in Corollary \ref{gencoro} is not necessarily
the minimal linear recurrence with integer coefficients satisfied
by $\{S(n)\}$. However, the characteristic polynomial of the minimal
of such recurrences must be a factor of $P_{S}(X)$.
\begin{example}
Let $F$ be a $n$-variable Boolean function. The \textit{nega-Hadamard
transform of F} is defined as the complex valued function given by
where $\Phi_{n}(X)$ is the $n$-th cyclotomic polynomial.
\end{example}

We would like to point out that this is not a new result. It was already
established in \cite{cms}. However, we decided to include it because
it is a straightforward application of our results. 
\begin{example}
Consider the sum The fact that $\mu_{S}(X)|P_{S}(X)$ is now evident. 
\end{example}

M
\begin{example}
Other toy examples can be constructed with previous classical results.
For example, it is known that $\{f_{n}\mod m\}$, where $f_{n}$ represents
the $n$-th Fibonacci number and $m$ is a positive integer, is periodic.
The period is known as the Pisano period mod $m$ and it is usually
denoted by $\pi(m)$. Let $f_{n}^{(m)}$ represent $f_{n}\mod m$
and consider the sum This example can be easily generalized to any
Lucas sequence of the first kind $u_{n}(a,b)$ (the Fibonacci sequence
is given by $u_{n}(1,-1)$). 
\end{example}

is of the same type as (\ref{gensum}). It remains to show the periodicity
of $F_{k;\mathbb{F}_{q}}$.

We start with the following lemma.
\begin{lem}
\label{lambdaperiod} Let $p$ be prime and $a_{1},\cdots,a_{l}$
be some elements in some field extension of $\mathbb{F}_{p}$. Define
\begin{equation}
\Lambda_{a_{1},\cdots,a_{l}}^{(p)}(k,m_{1},\cdots,m_{l})=\Lambda_{a_{1},\cdots,a_{l}}(k,m_{1}^{+},\cdots,m_{l}^{+})\mod p,
\end{equation}
where Then, $\Lambda_{a_{1},\cdots,a_{l}}^{(p)}(k,m_{1},\cdots,m_{l})$
is periodic in each of the variables $m_{1},\cdots,m_{l}$ with period
$D=p^{\lfloor\log_{p}(k)\rfloor+1}$. 
\end{lem}

We now present our closed formulas for $S_{\mathbb{F}_{q}}(\boldsymbol{e}_{n,k})$.
This generalizes Cai et al.'s result for the binary case \cite{cai}.
It also generalizes the recurrence exploited in \cite{cm1,cm2}.
\begin{thm}
\label{closedformsSq} Let $n$ and $k>1$ be positive integers and
$p$ be a prime and $q=p^{r}$ with $r\geq1$. Let $D=p^{\lfloor\log_{p}(k)\rfloor+1}$.
Then, 
\[
S_{\mathbb{F}_{q}}(\boldsymbol{e}_{n,k})=\sum_{j_{1}=0}^{D-1}\sum_{j_{2}=0}^{j_{1}}\cdots\sum_{j_{q-1}=0}^{j_{q-2}}c_{j_{1},\cdots,j_{q-1}}(k)\left(1+\xi_{D}^{-j_{1}}+\cdots+\xi_{D}^{-j_{q-1}}\right)^{n},
\]
$\xi_{m}=\exp(2\pi i/m)$, In particular, the sequence $\{S_{\mathbb{F}_{q}}(\boldsymbol{e}_{n,k})\}$
satisfies the linear recurrence with integer coefficients whose characteristic
polynomial is given by 
\[
P_{q,k}(X)=\prod_{a_{1}=0}^{D-1}\,\prod_{0\leq a_{2}\leq a_{1}}\cdots\prod_{0\leq a_{q-1}\leq a_{q-2}}\left(X-\left(1+\xi_{D}^{a_{1}}+\cdots+\xi_{D}^{a_{q-1}}\right)\right).
\]
\end{thm}

Theorem \ref{closedformsSq} also provides a bound for the degree
of the minimal linear recurrence with integer coefficients satisfied
by $\{S_{\mathbb{F}_{q}}(\boldsymbol{e}_{n,k})\}$.
\begin{cor}
Let $k>1$ be positive integers and $p$ be a prime and $q=p^{r}$
with $r\geq1$. Let $D=p^{\lfloor\log_{p}(k)\rfloor+1}$. The degree
of the minimal linear recurrence with integer coefficients that $\{S_{\mathbb{F}_{q}}(\boldsymbol{e}_{n,k})\}$
satisfies is less than or equal to $(D)_{q}/q!$, where $(a)_{n}=a(a+1)(a+2)\cdots(a+n-1)$
is the Pochhammer symbol. 
\end{cor}

\begin{example}
\label{exF4} Consider the sequence $\{S_{\mathbb{F}_{4}}(\boldsymbol{e}_{n,3})\}$.
Theorem \ref{closedformsSq} implies that this sequence satisfies
the linear recurrence whose characteristic is given by 
\begin{align*}
F_{8}({\bf X})= & X_{2}X_{3}X_{5}+X_{1}X_{4}X_{5}+X_{2}X_{4}X_{5}+X_{2}X_{7}X_{5}+X_{4}X_{7}X_{5}+X_{1}X_{8}X_{5}+X_{2}X_{8}X_{5}+\\
 & X_{3}X_{8}X_{5}+X_{4}X_{8}X_{5}+X_{1}X_{3}X_{6}+X_{2}X_{3}X_{6}+X_{1}X_{4}X_{6}+X_{2}X_{3}X_{7}+X_{1}X_{4}X_{7}+\\
 & X_{2}X_{4}X_{7}+X_{1}X_{6}X_{7}+X_{2}X_{6}X_{7}+X_{3}X_{6}X_{7}+X_{4}X_{6}X_{7}+X_{1}X_{3}X_{8}+X_{2}X_{3}X_{8}+\\
 & X_{1}X_{4}X_{8}+X_{1}X_{6}X_{8}+X_{3}X_{6}X_{8}.
\end{align*}
Observe that $S_{\mathbb{F}_{4}}(\boldsymbol{e}_{4,3})=S_{\mathbb{F}_{2}}(F_{8})=64.$ 
\end{example}

M
\begin{example}
onsider the sequence $\{S_{\mathbb{F}_{8}}(\boldsymbol{e}_{n,3})\}$.
Theorem \ref{closedformsSq} implies that this sequence satisfies
the linear recurrence whose characteristic is given by 
\begin{eqnarray*}
P_{8,3}(X) & = & \prod_{a_{1}=0}^{3}\prod_{a_{2}=0}^{a_{1}}\prod_{a_{3}=0}^{a_{2}}\prod_{a_{4}=0}^{a_{3}}\prod_{a_{5}=0}^{a_{4}}\prod_{a_{6}=0}^{a_{5}}\prod_{a_{7}=0}^{a_{6}}\left(X-\left(1+i^{a_{1}}+i^{a_{2}}+i^{a_{3}}+i^{a_{4}}+i^{a_{5}}+i^{a_{6}}+i^{a_{7}}\right)\right).
\end{eqnarray*}
The minimal linear recurrence with integer coefficients that $\{S_{\mathbb{F}_{8}}(\boldsymbol{e}_{n,3})\}$
satisfies has characteristic polynomial given by As with the previous
can be identified with a $3n$-variable Boolean function. 
\end{example}

These two examples show a big difference between the degrees of the
polynomials $P_{q,k}(X)$ and $\mu_{q,k}(X)$, where $\mu_{q,k}(X)$
represents the characteristic polynomial of the minimal linear recurrence
with integer coefficients satisfied by the sequence $\{S_{\mathbb{F}_{q}}(\boldsymbol{e}_{n,k})\}$.
In particular, $P_{q,k}(X)$ does not seem to be tight. However, what
you are seeing here is the fact that when working over $\mathbb{F}_{q}$
with $q=p^{r}$ and $r>1$, some of the factors of $P_{q,k}(X)$ are
repeated multiple times. For instance, consider Example \ref{exF4}.
Observe that when $(a_{1},a_{2},a_{3})=(2,1,0)$ we get the factor
$X-(1+i)$. However when $(a_{1},a_{2},a_{3})=(3,1,1)$, we also get
the factor $X-(1+i)$. Therefore, this factor is repeated twice. The
factor $X-(1-i)$ is also repeated twice. That is why the factor $X^{2}-2X+2$
appears in $P_{4,3}(X)$ with 2 as exponent. This phenomenon does
not occur over $\mathbb{F}_{p}$. In fact, there are examples where
the polynomial $P_{p,k}(X)$ is tight.
\begin{example}
Consider the sequence $\{S_{\mathbb{F}_{3}}(\boldsymbol{e}_{n,7})\}$.
The characteristic polynomial of the minimal linear recurrence with
integer coefficients satisfied by this sequence is 
\[
\mu_{3,7}(X)=\frac{1}{X}P_{3,7}(X).
\]
The te
\end{example}

The repetition of factors can be eliminated by using \textit{least }
\begin{thm}
Let $n$ and $k>1$ be positive integers and $p$ be a prime and $q=p^{r}$
with $r\geq1$. Let $D=p^{\lfloor\log_{p}(k)\rfloor+1}$. Let $M_{a_{1},\cdots,a_{q-1}}(X)$
be the minimal polynomial for the algebraic integer $1+\xi_{D}^{a_{1}}+\cdots+\xi_{D}^{a_{q-1}}$.
Then, $\{S_{\mathbb{F}_{q}}(\boldsymbol{e}_{n,k})\}$ satisfies the
linear recurrence with integer coefficients whose characteristic polynomial
is given by 
\end{thm}

\begin{rem}
As expected, having these recurrences at hand allow us to compute
exponential sums of elementary symmetric polynomials for big values
of $n$. For instance, it took \textit{Mathematica} 37.504 seconds
to calculate $S_{\mathbb{F}_{3}}(\boldsymbol{e}_{100,000,11})$, which
is a integer with $47,712$ digits.
\end{rem}

We point out that Theorem \ref{closedformsSq} and other results after
it can be extended to linear combinations of elementary symmetric
polynomials without too much effort. For instance, suppose that $0\leq k_{1}<\cdots<k_{s}$
are integers and $\beta_{1},\cdots,\beta_{s}\in\mathbb{F}_{q}^{\times}$.
The discussion prior Theorem \ref{closedformsSq} together with Corollary
\ref{expsumformcoro} implies that The statement of Theorem \ref{closedformsSq}
can now be written almost verbatim for linear combinations of elementary
symmetric polynomials. The only differences are that $D$ is now $D=p^{\lfloor\log_{p}(k_{s})\rfloor+1}$
and Similar adjustments apply to the other results.

%%%%%%%%%%%%%%%%%%%%%%%%%%%%%%%%%%%%%%%%%%%%%%%%%%%%
% Preliminaries
%%%%%%%%%%%%%%%%%%%%%%%%%%%%%%%%%%%%%%%%%%%%%%%%%%%%

\section{Concluding remarks}

We
\begin{thebibliography}{10}
\bibitem{sperber} A. Adolphson and S. Sperber. \newblock $p$-adic
Estimates for Exponential Sums and the of Chevalley-Warning. \newblock
\textit{Ann. Sci. Ec. Norm. Super.}, $4^{e}$ s�rie, \textbf{20},
545\textendash 556, 1987.

\bibitem{acgmr} R. A. Arce-Nazario, F. N. Castro, O. E. Gonz�lez,
L. A. Medina and I. M. Rubio. \newblock New families of balanced
symmetric functions and a generalization of Cuscik, Li and P. St$\check{\mbox{a}}$nic$\check{\mbox{a}}$.
\newblock \textit{Designs, Codes and Cryptography} \textbf{86}, 693\textendash 701,
2018.

\bibitem{ax} J. Ax. \newblock Zeros of polynomials over finite fields.
\newblock \textit{Amer. J. Math.}, \textbf{86}, 255\textendash 261,
1964.

\bibitem{BCP} M. L. Bileschi, T.W. Cusick and D. Padgett. \newblock
Weights of Boolean cubic monomial rotation symmetric functions. \newblock
\textit{Cryptogr. Commun.}, \textbf{4}, 105\textendash 130, 2012.

\bibitem{browncusick} A. Brown and T. W. Cusick. \newblock Recursive
weights for some Boolean functions. \newblock \textit{J. Math. Cryptology},
\textbf{6(2)}, 105\textendash 135, 2012.

\bibitem{cai} J. Cai, F. Green and T. Thierauf. \newblock On the
correlation of symmetric functions. \newblock \textit{Math. Systems
Theory}, \textbf{29}, 245\textendash 258, 1996.

\bibitem{canteaut} A. Canteaut and M. Videau. \newblock Symmetric
Boolean Functions. \newblock \textit{IEEE Trans. Inf. Theory} \textbf{51(8)},
2791\textendash 2881, 2005.

\bibitem{Davis} \newblock Philip Davis. Circulant Matrices. \newblock
Chelsea publishing, Second Edition,1994.

\bibitem{cmg3} F. N. Castro, O. E. Gonz�lez and L. A. Medina. \newblock
Diophantine Equations With Binomial Coefficients and Perturbations
of Symmetric Boolean Functions. \newblock \textit{IEEE Trans. Inf.
Theory}, \textbf{64(2)}, 1347\textendash 1360, 2018.

\bibitem{cm1} F. N. Castro and L. A. Medina. \newblock Linear Recurrences
and Asymptotic Behavior of Exponential Sums of Symmetric Boolean Functions.
\newblock \textit{Elec. J. Combinatorics}, 18:\#P8, 2011.

\bibitem{cm2} F. N. Castro and L. A. Medina. \newblock Asymptotic
Behavior of Perturbations of Symmetric Functions. \newblock \textit{Annals
of Combinatorics}, 18:397\textendash 417, 2014.

\bibitem{cm3} F. N. Castro and L. A. Medina. \newblock Modular periodicity
of exponential sums of symmetric Boolean functions. \newblock \textit{Discrete
Appl. Math.} \textbf{217}, 455\textendash 473, 2017.

\bibitem{cms} F. N. Castro, L. A. Medina and P. St$\check{\mbox{a}}$nic$\check{\mbox{a}}$.
\newblock Generalized Walsh transforms of symmetric and rotation
symmetric Boolean functions are linear recurrent. \newblock \textit{Appl.
Algebra Eng. Commun. Comput.}, DOI 10.1007/s00200-018-0351-5, 2018.

\bibitem{ccms} F. N. Castro, R. Chapman, L. A. Medina, and L. B.
Sep�lveda. \newblock Recursions associated to trapezoid, symmetric
and rotation symmetric functions over Galois fields. \newblock to
appear in Discrete Math.

\bibitem{cusick4} T. W. Cusick. \newblock Hamming weights of symmetric
Boolean functions. \newblock \textit{Discrete Appl. Math.} \textbf{215},
14\textendash 19, 2016.

\bibitem{cusickArXiv} T. W. Cusick. \newblock Weight recursions
for any rotation symmetric Boolean functions. \newblock \textit{IEEE
Trans. Inf. Theory}, \textbf{64}, 2962 - 2968, 2018.

\bibitem{cusickjohns} T. W. Cusick and B. Johns. \newblock Recursion
orders for weights of Boolean cubic rotation symmetric functions.
\newblock \textit{Discr. Appl. Math.}, \textbf{186}, 1\textendash 6,
2015.

\bibitem{cusick2} T. W. Cusick, Y. Li, and P. St$\check{\mbox{a}}$nic$\check{\mbox{a}}$.
\newblock Balanced Symmetric Functions over $GF(p)$. \newblock
\textit{IEEE Trans. Inf. Theory}, \textbf{5}, 1304\textendash 1307,
2008.

\bibitem{cusick2.1} T. W. Cusick, Y. Li, and P. St$\check{\mbox{a}}$nic$\check{\mbox{a}}$.
\newblock On a conjecture for balanced symmetric Boolean functions.
\newblock \textit{J. Math. Crypt.}, \textbf{3}, 1\textendash 18,
2009.

\bibitem{cusickstanica} T.W. Cusick and P. St$\check{\mbox{a}}$nic$\check{\mbox{a}}$.
\newblock Fast evaluation, weights and nonlinearity of rotation symmetric
functions. \newblock \textit{Discr. Math.}, \textbf{258}, 289\textendash 301,
2002.

\bibitem{dalaimaitrasarkar} D. K. Dalai, S. Maitra and S. Sarkar.
\newblock Results on rotation symmetric Bent functions. \newblock
\textit{Second International Workshop on Boolean Functions: Cryptography
and Applications, BFCA'06}, publications of the universities of Rouen
and Havre, 137\textendash 156, 2006.

\bibitem{fengliu} K. Feng and F. Liu. \newblock New Results On The
Nonexistence of Generalized Bent Functions. \newblock \textit{IEEE
Trans. Inf. Theory} \textbf{49}, 3066\textendash 3071, 2003.

\bibitem{ggz} Y. Guo, G. Gao, Y. Zhao. \newblock Recent Results
on Balanced Symmetric Boolean Functions. \newblock \textit{IEEE Trans.
Inf. Theory} \textbf{62 (9)}, 5199\textendash 5203, 2016.

\bibitem{hell} M. Hell, A. Maximov and S. Maitra. \newblock On efficient
implementation of search strategy for rotation symmetric Boolean functions.
\newblock \textit{Ninth International Workshop on Algebraic and Combinatorial
Coding Theory, ACCT 2004}, Black Sea Coast, Bulgaria, 2004.

\bibitem{hg} Y. Hu and G. Xiao. \newblock Resilient Functions Over
Finite Fields. \newblock \textit{IEEE Trans. Inf. Theory} \textbf{49},
2040\textendash 2046, 2003.

\bibitem{fspectrum} M. Kolountzakis, R. J. Lipton, E. Markakis, A.
Metha and N. K. Vishnoi. \newblock On the Fourier Spectrum of Symmetric
Boolean Functions. \newblock \textit{Combinatorica}, \textbf{29},
363\textendash 387, 2009.

\bibitem{KScW} P.V. Kumar, R.A. Scholtz, and L.R. Welch. \newblock
Generalized Bent Functions and Their Properties. \newblock \textit{J.
Combinatorial Theory (A)}, \textbf{40}, 90\textendash 107, 1985.

\bibitem{licusick1} Y. Li and T.W. Cusick. \newblock Linear Structures
of Symmetric Functions over Finite Fields. \newblock {Inf. Processing
Letters} \textbf{97}, 124\textendash 127, 2006.

\bibitem{licusick2} Y. Li and T. W. Cusick. \newblock Strict Avalanche
Criterion Over Finite Fields. \newblock \textit{J. Math. Cryptology}
\textbf{1(1)}, 65\textendash 78, 2006.

\bibitem{llm}M. Liu, P. Lu and G.L. Mullen. \newblock Correlation-Immune
Functions over Finite Fields. \newblock \textit{IEEE Trans. Inf.
Theory} \textbf{44}, 1273\textendash 1276, 1998.

\bibitem{maxhellmaitra} A. Maximov, M. Hell and S. Maitra. \newblock
Plateaued Rotation Symmetric Boolean Functions on Odd Number of Variables.
\newblock \textit{First Workshop on Boolean Functions:Cryptography
and Applications, BFCA'05}, publications of the universities of Rouen
and Havre, 83\textendash 104, 2005.

\bibitem{mitchell} C. Mitchell. \newblock Enumerating Boolean functions
of cryptographic significance. \newblock {\em J. Cryptology} \textbf{2}
(3), 155\textendash 170, 1990.

\bibitem{mm1} O. Moreno and C. J. Moreno. \newblock Improvement
of the Chevalley-Warning and the Ax-Katz theorems. \newblock \textit{Amer.
J. Math.}, \textbf{117}, 241\textendash 244, 1995.

\bibitem{mm} O. Moreno and C. J. Moreno. \newblock The MacWilliams-Sloane
Conjecture on the Tightness of the Carlitz-Uchiyama Bound and the
Weights of Dual of BCH Codes. \newblock \textit{IEEE Trans. Inform.
Theory}, \textbf{40}, 1894\textendash 1907, 1994.

\bibitem{parkerpott} M. G. Parker and A. Pott. \newblock On Boolean
functions which are bent and negabent. \newblock \textit{Proc. Int.
Conf. Sequences, Subsequences, Consequences}, LNCS-4893, 9\textendash 23,
2007.

\bibitem{piequ} J. Pieprzyk and C.X. Qu. \newblock Fast hashing
and rotation-symmetric functions. \newblock \textit{J. Universal
Comput. Sci.}, \textbf{5 (1)}, 20\textendash 31, 1999.

\bibitem{rp} C. Riera and M. G. Parker. \newblock Generalized bent
criteria for Boolean functions. \newblock \textit{IEEE Trans. Inform.
Theory} \textbf{52} (9), 4142\textendash 4159, 2006.

\bibitem{fdegree} A. Shpilka and A. Tal. \newblock On the Minimal
Fourier Degree of Symmetric Boolean Functions. \newblock \textit{Combinatorica},
\textbf{88}, 359\textendash 377, 2014.

\bibitem{stanicamaitra} P. St$\check{\mbox{a}}$nic$\check{\mbox{a}}$
and S. Maitra. \newblock Rotation Symmetric Boolean Functions \textendash{}
Count and Cryptographic Properties. \newblock \textit{Discr. Appl.
Math.}, \textbf{156}, 1567\textendash 1580, 2008

\bibitem{stanicamaitraclark} P. St$\check{\mbox{a}}$nic$\check{\mbox{a}}$,
S. Maitra and J. Clark. \newblock Results on Rotation Symmetric Bent
and Correlation Immune Boolean Functions. \newblock \textit{Fast
Software Encryption, FSE 2004}, Lecture Notes in Computer Science,
\textbf{3017}, 161\textendash 177. SpringerVerlag, 2004.
\end{thebibliography}

\end{document}
